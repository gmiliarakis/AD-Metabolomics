%%%%%%%%%-----------------------------------------------------------------
%%%%%%%%% Thesis template, v1
%%%%%%%%% Mathematical & Statistical Methods group - Biometris
%%%%%%%%% Wageningen University & Research
%%%%%%%%%-----------------------------------------------------------------


\documentclass{amsart}
%  Font & Formatting
\usepackage{lmodern}
\usepackage[x11names]{xcolor}

%  Text encoding
\usepackage[utf8]{inputenc}
\usepackage{booktabs}
% \usepackage{lipsum}
\setcounter{tocdepth}{3}

% Setting margins
\usepackage[a4paper, left=3cm,right=3cm]{geometry}

% Packages for the titlepage
\usepackage{tikz}
\usetikzlibrary{calc}
\usepackage{graphicx}
\usepackage{newtxtext}
\usepackage{float}
\usepackage[inkscapeformat=png]{svg}

% Packages for tables
\usepackage{multicol}
\usepackage{multirow}
\usepackage{array}
\usepackage{ragged2e}
\usepackage{colortbl}
\usepackage{threeparttable}

% Packages for mathematical writing
\usepackage{amsmath}                           % Enables the align enviroment
\usepackage{amssymb}                           % Math symbols (e.g. \mathbb{})
\usepackage{dsfont} 	                         % For \mathds{1} blackboard bold 1
\usepackage{bm}                                % For roman (upright) bold latin letters
\usepackage{mathtools}                         % Better math

% For algorithms
\usepackage[ruled,vlined,linesnumbered]{algorithm2e}
% \usepackage{algpseudocode}
% \renewcommand{\algorithmicrequire}{\textbf{Input:}}
% \renewcommand{\algorithmicensure}{\textbf{Output:}}

% For urls and hyperlinks
\usepackage[scientific-notation=true, allow-number-unit-breaks=true]{siunitx} % For scientific notation
\usepackage{euscript}[mathcal]
\usepackage{txfonts}
\usepackage[
  hypertexnames=false,
  colorlinks= true,
  linkcolor=.,
  citecolor={Turquoise4},
  urlcolor={PaleTurquoise4}]{hyperref} 

% For bibliography
\usepackage[backend=biber,
    style=nature, 
    maxcitenames=1,
    mincitenames=1,
    safeinputenc]
    {biblatex}
\addbibresource{literature/references.bib}
\def\bibfont{\footnotesize}

% Abbreviations
\usepackage[acronym, nomain, nonumberlist]{glossaries}
\makeglossaries
\newacronym{ad}{AD}{Alzheimer's disease}
\newacronym{sad}{SAD}{sporadic Alzheimer's disease}
\newacronym{fad}{FAD}{familial Alzheimer's disease}
\newacronym{apoe}{ApoE}{apolipoprotein E}
\newacronym{load}{LOAD}{late-onset Alzheimer's disease}
\newacronym{idl}{IDL}{intermediate density lipoprotein}
\newacronym{vldl}{VLDL}{very low density lipoprotein}
\newacronym{abca1}{ABCA1}{ATP-binding cassette transporter A1}
\newacronym{lrp}{LRP1}{low density lipoprotein receptor-related protein 1}
\newacronym{hdl}{HDL}{high density lipoprotein}
\newacronym{bbb}{BBB}{blood-brain barrier}
\newacronym{tlr4}{TLR4}{toll-like receptor 4}
\newacronym{nfkb}{NFkB}{nuclear factor kappa B}
\newacronym{atp}{ATP}{adenosine tri-phosphate}
\newacronym{vcs}{VCS}{version control system}
\newacronym{cv}{CV}{cross-validation}
\newacronym{gsk}{GSK-3$\beta$}{glycogen synthase kinase-3$\beta$}
\newacronym{csf}{CSF}{cerebrospinal fluid}
\newacronym{rss}{RSS}{residual sum of squares}
\newacronym{ancova}{ANCOVA}{analysis of covariance}
\newacronym{smote}{SMOTE}{synthetic minority over-sampling technique}
\newacronym{ml}{ML}{maximum likelihood}
\newacronym{lasso}{LASSO}{least absolute shrinkage and selection operator}
\newacronym{mnl}{MLR}{multinomial logistic regression}
\newacronym{roc}{ROC}{receiver-operating characteristics}
\newacronym{auc}{AUC}{area under the ROC curve}
\newacronym{ggm}{GGM}{Gaussian graphical model}
\newacronym{app}{APP}{amyloid precursor protein}
\newacronym{ldlr}{LDLR}{low-density lipoprotein receptor}
\newacronym{isf}{ISF}{interstitial fluid}
\newacronym{il}{IL}{interleukin}
\newacronym{ros}{ROS}{reactive oxygen species}
\newacronym{tnf}{TNF-$\alpha$}{tumor necrosis factor $\alpha$}
\newacronym{xgb}{XGBoost}{eXtreme gradient boosting}
\newacronym{dt}{DT}{decision tree}
\newacronym{mci}{MCI}{mild cognitive impairment}
\newacronym{scd}{SCD}{subjective cognitive decline}
\renewcommand{\glossarymark}[1]{}


%--------------------------------------------------------------------
%--------------- Front and Main matter style ------------------------
\newcommand{\frontmatter}{
    \pagenumbering{roman}   % Setting page numbering to lower-case roman
}
\newcommand{\mainmatter}{
    \newpage
    \pagenumbering{arabic}  % Setting page numbering to normal integers
}
%--------------- Front and Main matter style ------------------------
%--------------------------------------------------------------------

\begin{document}
% \Sconcordance{concordance:Thesis_Template_Biometris.tex:Thesis_Template_Biometris.rnw:1 %
65 1 1 0 345 1 1 9 11 0 1 2 136 1}



% Add title page:
%------------------------------------------------------------------------------
% In this segment, enter the desired student data to be shown at the title page
\newcommand{\thesisAuthor}{George Miliarakis}                             % State your name
\newcommand{\thesisTitle}{ApoE4 dose effects on serum metabolites in Alzheimer's disease}                                      % State title thesis
\newcommand{\thesisSubTitle}{mechanistic insights from a data science approach}                            % State subtitle thesis
\newcommand{\thesisDegree}{MSc Thesis}      % Choose type
\newcommand{\university}{Wageningen University \& Research}                % You generally do not need to touch this
\newcommand{\thesisPlaceDate}{Wageningen, March 2024}                      % State month and year
%------------------------------------------------------------------------------


%------------------------------------------------------------------------------
% In this segment, enter the desired supervisor data to be shown at the title page
\newcommand{\supervisor}{C.F.W. Peeters}                                                % State name of supervisor
\newcommand{\departmentSUP}{Mathematical \& Statistical Methods (Biometris)} % State department supervisor (generally Biometris)
\newcommand{\universitySUP}{Wageningen University \& Research}                              % State university supervisor (generally WUR)
%------------------------------------------------------------------------------


%------------------------------------------------------------------------------
% In this segment, enter the desired co-supervisor data to be shown at the title page
\newcommand{\cosupervisor}{Yannick Vermeiren}                                         % State name of co-supervisor
\newcommand{\departmentCOSUP}{Nutritional Biology, Human Nutrition \& Health}                             % State department or division co-supervisor
\newcommand{\universityCOSUP}{Wageningen University \& Research}                              % State university or company co-supervisor
%------------------------------------------------------------------------------


%------------------------------------------------------------------------------
% Logos and visuals
\begin{titlepage}
\thispagestyle{empty}

% Adding logos
\begin{figure} [H]
\vspace{-3cm}
 \centering
\begin{minipage}[t]{.45\linewidth}
  \raggedright
  % Upload and include SLU loggo here:
  \hspace*{-2cm}\includegraphics[width=\linewidth]{figures/WURlogo.png}

\end{minipage}%
  \begin{minipage}[t]{.45\linewidth}
  \vspace{-2.6cm}
 \raggedleft
%% Inclusion Biometris logo
 \hspace*{+2cm}\includegraphics[width =.95\textwidth]{figures/biometris_logo.png}

\end{minipage}
\end{figure}

% Bottom/background picture
\begin{tikzpicture}[overlay, remember picture]
\node[anchor=south west,
      xshift=+6.5cm,
      yshift=-0.2cm]
     at (current page.south west)
     {\includegraphics[width = 0.5\textwidth]{figures/torontodeclaration.png}};
\end{tikzpicture}
%------------------------------------------------------------------------------


%------------------------------------------------------------------------------
% Title information
\vspace{1cm}
\begin{center}
\par
\noindent
\rule[0.2cm]{\linewidth}{1.5pt}
\Huge
\textbf{\thesisTitle}
\vspace{0.2cm}
\LARGE
\par
\noindent
\thesisSubTitle\\
\rule[0.2cm]{\linewidth}{1.5pt}
\Large
\end{center}
%------------------------------------------------------------------------------


%------------------------------------------------------------------------------
% Author information
\vspace{2cm}
\noindent
\LARGE
\thesisAuthor\\
\vspace{.2 cm}
\small
\par \noindent
\thesisDegree
\par \noindent
\university
\par \noindent
\thesisPlaceDate
%------------------------------------------------------------------------------


%------------------------------------------------------------------------------
% Supervision information
\vspace{4cm}
\begin{flushright}
\emph{Supervisor} \\
\textbf{\supervisor} \\
\departmentSUP\\
\universitySUP
\end{flushright}

\vspace{.5cm}
\begin{flushright}
\emph{Supervisor} \\
\textbf{\cosupervisor} \\
\departmentCOSUP \\
\universityCOSUP
\end{flushright}
\end{titlepage}
%------------------------------------------------------------------------------
%--------------- Front matter ---------------------------------------
%-------------------------------------------------------------------
\frontmatter
% Foreword
% \section*{Foreword}
% This is where you type your foreword.
% In the Foreword the student states clearly his/her contribution that originates from the master thesis project, and, if applicable, what contribution(s) possibly followed from the student’s internship on the same topic.
% It is also stated in the foreword what data sources were used, and whether the data have a degree of confidentiality.
% In addition, possible acknowledgements may be made to people who have contributed to (certain parts of) the thesis.


% % Abstract
% \newpage
% \section*{Abstract}
% This is where you type your abstract.
% The abstract must be communicable to a broad audience and contain, next to a summary text, an outreach item such as an infographic or a link to a video/website/web application.
% Such as, for example, the nice figure below.


% Glossary
\clearpage
\printacronyms[title = Abbreviations, toctitle = Abbreviations]

\newpage
% Inserting table of contents
\tableofcontents

%--------------------------------------------------------------------
%--------------- Main matter ----------------------------------------
\mainmatter

\newpage
\section{Introduction}\label{Intro}
\subsection{Alzheimer’s disease}
\acrfull{ad} is a complex, progressive neurodegenerative disorder and the most common form of dementia \cite{Penke2023NewDisease}. It was considered the 6th leading cause of death in the US in 2019, with an overall increase of 145\% in mortality from 2000 to 2019 \cite{20232023Figures}. The impact AD has on patients, their caregivers, and healthcare systems is detrimental. Hence, a considerable amount of research has been performed in an effort to understand, prevent, impede or cure it. Nonetheless, important aspects of its systemic manifestations remain unknown.

Age is the main risk factor for AD, while several genetic and lifestyle risk factors, as well as biochemical pathways contribute to its development \cite{Penke2023NewDisease}. AD manifests in various histopathological phenotypes and presents a broad spectrum of clinical signs and symptoms \cite{Heneka2015NeuroinflammationDisease, Edwards2019ANeurodegeneration}. The AD continuum starts with \acrfull{scd}, followed by \acrfull{mci} \cite*{AALDIJK2022101556}, and continues with progressive loss of global cognition, of which particularly memory, processing speed and executive functioning, spanning a total period of 10-15 years \cite{Scheltens2016AlzheimersDisease}. 

Sporadic or late-onset AD (\acrshort{sad}, \acrshort{load}; \>95\% of cases) is the most frequent phenotype, typically appearing after 65 years of age \cite{Beydoun2014EpidemiologicMeta-analysis}. A rarer phenotype is early-onset familial AD (\acrshort{fad}), usually starting at ages 30–65 and passed in an autosomal dominant fashion \cite{VanCauwenberghe2015ThePerspectives}.
Even though FAD mutations explain only a small percentage of AD cases, they have a great impact on AD research given their appealing genotype-phenotype links.

\subsubsection{Amyloid cascade hypothesis}
The amyloid cascade hypothesis has dominated the scientific discussion on the pathogenesis of AD. In historical terms, its impact was profound, as it helped distinguish and identify AD as a single disease that may be studied for treatment \cite{Hardy2006AlzheimersReappraisal}. It suggests that chronic neuroinflammation promotes protein misfolding and accumulation in the brain, forming plaques (consisting of oligomerized amyloid A$\beta_{42}$) and tangles (consisting of hyperphosphorylated tau protein) \cite{Edwards2019ANeurodegeneration}. Nevertheless, it does not necessarily cover all AD cases. Clinical trials with A$\beta$ antibodies as a potential disease-modifying treatment have so far failed to demonstrate any clinical improvement in patients, despite successful removal of plaques from AD brains \cite{Kepp2023TheReview,Kurkinen2023TheThinking}. Evidence suggests that oxidative stress, metabolic abnormalities, atherosclerosis, cardiovascular effects, imbalances of intra-neuronal calcium and metal ion depositioning might also contribute to the development of AD \cite{Kepp2023TheReview}. \citeauthor{Kepp2023TheReview} propose a more complex and holistic view of AD pathology, by integrating (epi-)genetic, environmental, vascular, neuro-inflammatory and metabolic factors in disease-predicting models \cite{Kepp2023TheReview}.

\subsection{Human apolipoprotein-E gene}
LOAD has been associated with genetic risk factors, such as with the gene that encodes for \acrfull{apoe} \cite{Corder1993GeneFamilies}. The structure and function of ApoE, as well as how the first defines the latter is described in Section \ref{ApoEprot}.

\subsubsection{ApoE4 and Alzheimer's disease}
Humans due to two point mutations in the ApoE gene (rs7412 C/T, rs429358 C/T) present three common variants: $\varepsilon_2$, $\varepsilon_3$ and $\varepsilon_4$ \cite{Husain2021APOETherapeutics, Yang2023ApolipoproteinDisease}. The three ApoE haplotypes result in six genotypes [Fig. \ref{fig1}]. In Caucasian populations, the most abundant allele is ApoE3 (rs7412 C, rs429358 T), with a frequency of 78\%, and is considered neutral regarding the risk of \acrshort{ad} \cite{Liu2013ApolipoproteinTherapy}. ApoE4 (rs7412 C, rs429358 C) has a frequency of 14\% and represents the strongest genetic risk factor for LOAD, with gene dose effects \cite{Strittmatter1993ApolipoproteinDisease}. Conversely, ApoE2 (rs7412 T, rs429358 T) is found in 8\% of Caucasian populations, and is associated with a reduced risk of LOAD \cite{Liu2013ApolipoproteinTherapy}. The ApoE haplotypes are strongly linked to the primary pathological features of LOAD, namely A$\beta$ and phosphorylated tau \cite{Deming2017Genome-wideModifiers}. While the association between ApoE alleles and LOAD risk or protection is observed across diverse ancestral backgrounds, the strength of this association varies \cite{Belloy2019AForward, Farrer1997EffectsMeta-analysis}, see Section \ref{ancestry}.

Carrying a single $\varepsilon_4$ allele implies a risk of developing AD of about 20\% \cite{Bookheimer2009APOE4GA}. A heterozygous ApoE4 allelic composition (e.g., $\varepsilon_4$/$\varepsilon_x$) is associated with approximately 3-4 times increased life-time risk of developing AD, while the homozygous composition ($\varepsilon_4$/$\varepsilon_4$) 12 times increased \cite{Kim2009TheRO}. Even though ApoE4 is the strongest known genetic risk factor for LOAD, it is neither necessary nor sufficient to cause AD, and it is certainly not the only genetic risk factor \cite{SerranoPozo2019IsAD}.

\begin{figure}[H]
  \includegraphics[width=0.7\textwidth]{figures/ApoE@2x.png}
    \caption{Sankey chart reflecting the distribution of ApoE alleles in the composition of the 6 ApoE genotypes.}
  \label{fig1}
\end{figure}

\subsubsection{ApoE and sex synergy in Alzheimer's disease}
Notably, sex (60\% females) and ApoE4 allelic composition (50\% has at least one $\varepsilon_4$ allele) are the strongest genetic risk factors for \acrshort{load} \cite{Arnold2020SexMetabolome}. In this regard, it is shown that the ApoE4 genotype has a larger impact on females, as they present greater impairment of mitochondrial energy production, compared to males \cite{Arnold2020SexMetabolome, Yassine2020APOEDisease}.

\subsubsection{ApoE and ancestry in Alzheimer's disease} \label{ancestry}
The majority of studies exploring the relationship between ApoE alleles and the genetics of LOAD have primarily focused on Northern European populations\cite{Yang2023ApolipoproteinDisease}. However, smaller studies involving diverse ancestral backgrounds show variations in ApoE4 allele \cite{Yang2023ApolipoproteinDisease}. While ApoE4 is present in 14\% of Caucasian Americans, its prevalence increases to 40\% among African Americans, 37\% in Oceania, and 26\% in Australia. Southern Asia and Europe exhibit ApoE4 allelic frequencies of less than 10\%, compared to Northern Europe where it rises to 25\% \cite{Belloy2019AForward, Egert2012ApoEFactors, Eisenberg2010WorldwideHistory, Logue2011AAmericans}.

ApoE alleles have different impact on distinct populations. ApoE4 implies a higher risk of LOAD for Korean, Japanese, and Japanese-American, compared to Caucasian populations \cite{Farrer1997EffectsMeta-analysis}. Conversely, ApoE4 is associated with a lower risk of LOAD in Native Americans, Hispanic Americans, African Americans, and those of African descent, compared to Caucasian-American populations \cite{Farrer1997EffectsMeta-analysis, Blue2019LocalHispanics, Suchy-Dicey2022APOEStudy, Rajabli2018AncestralPopulations, Naslavsky2022GlobalSample}. Notably, ApoE3 is more protective than ApoE2 against AD, as reported by a study in a Chinese population \cite{Chen2011ApolipoproteinDisease}. Some of these population-specific effects are attributed to the ApoE haplotype and ancestral variations in the ApoE gene beyond the $\varepsilon_2/\varepsilon_3/\varepsilon_4$ haplotypes \cite{Blue2019LocalHispanics, Rajabli2018AncestralPopulations}.


\subsubsection{Evolution of ApoE over time}
Interestingly, humans are the only species exhibiting polymorphism in the ApoE gene \cite{Yassine2020APOEDisease}. All other animal species have one ApoE variant, which resembles the human ApoE3 allele \cite{Hunsberger2019TheInterventions}. ApoE4 is the oldest human allele, followed by ApoE3 and ApoE2 in age \cite{Yassine2020APOEDisease}. ApoE4 may be adaptive, reducing mortality in highly infectious environments, with food scarcity and shorter lifespans \cite{Trumble2017ApolipoproteinBurden}. However, as human environments became less septic, with food abundance and longer life expectancy, ApoE4 started to burden the arteries and brain, increasing the risk of diseases related to ageing \cite{Yassine2020APOEDisease}. The emergence of ApoE3 from ApoE4 putatively reflects the shift in human diet from a plant-based one to a meat-rich one, where genes adaptive to high meat consumption were, and still are vital to regulate increased cholesterol levels \cite{Finch1999TheIsoforms}. 

\subsubsection{Tissue expression}
The principal producers of ApoE are hepatocytes in the liver \cite{Mahley2016CentralMetabolism}. In the CNS, ApoE is primarily expressed in glia (non-neuronal cells of the brain that support the neurons) and particularly in astrocytes (metabolic homeostasis and neuronal communication modulators), followed by microglia (the brain immune cells) \cite{Lanfranco2021ExpressionInflammation}. Each genotype is linked to different expression levels \cite{Husain2021APOETherapeutics}. ApoE2 carriers seem to have higher \acrfull{csf} levels of ApoE, compared to ApoE4 carriers \cite{Castellano2011HumanClearance, Cruchaga2012CerebrospinalDisease}. 

\subsection{Apolipoprotein E}\label{ApoEprot}
\subsubsection{Structure and function}
\acrshort{apoe} is a brain-specific lipid-binding glycoprotein of 299 amino-acids (34 kDa) that comprises several types of lipoproteins, i.e., chilomicra, \acrlong{idl} and \acrlong{vldl} \cite{Husain2021APOETherapeutics}. Its main function in the brain is the transport of lipids (mainly cholesterol) through membrane receptors \cite{Yang2023ApolipoproteinDisease}. Moreover, its isoforms have an effect on diverse cellular functions, e.g. synaptic integrity, glucose metabolism, A$\beta$ clearance, \acrlong{bbb} integrity and mitochondrial regulation \cite{Husain2021APOETherapeutics}. How these relate to \acrshort{ad} pathology will be elaborated in Section \ref{ApoEAD}.



\begin{figure}[htb]
  \includegraphics[width=0.8\textwidth]{figures/ApoEprot.png}
    \caption{Linear representation of the ApoE protein. Three structural domains are highlighted: N-terminal, hinge and C-terminal domains. The different amino acids at positions 112 and 158 are shown per common alleles and amino acids at positions 136, 236 and 251 coded by rarer alleles. Source: \citetitle{Bu2022APOEVariants}, \Citeauthor{Bu2022APOEVariants} (\citeyear{Bu2022APOEVariants}) \cite{Bu2022APOEVariants}}
  \label{fig2}
\end{figure}

\subsubsection{ApoE isoforms}
The nuances in the amino acid composition of ApoE, specifically the presence of cysteine or arginine at positions 112 and 158, significantly impact its binding with lipids and receptors \cite{Yassine2020APOEDisease}. The most prevalent isoform, ApoE3, features cysteine at position 112 and arginine at position 158  \cite{Yassine2020APOEDisease}, as shown in Fig. \ref{fig2}. ApoE2 has two cysteines, while ApoE4 has two arginines at these positions \cite{Yassine2020APOEDisease}. The C-terminal domain of ApoE (positions 273–299) is crucial for its lipidation specificity and efficiency \cite{Hu2015OpposingMice}.

\subsubsection{Lipidation nuances of ApoE isoforms}
For ApoE to exert its effects, it needs to be lipidated \cite{Husain2021APOETherapeutics}. ApoE undergoes lipidation via \acrfull{abca1}, a lipid efflux protein \cite{Flowers2020APOEBrain, Courtney2016LXRDisease}. The lipidation degree varies among ApoE isoforms, with ApoE4 exhibiting the least efficient lipidation \cite{Hu2015OpposingMice, Heinsinger2016ApolipoproteinFluid}. This discrepancy in lipidation has been linked to alterations in lipoprotein size and type, in that ApoE4 "prefers" large triglyceride-rich VLDL, while ApoE2 and ApoE3 have a higher affinity for phospholipid-rich \acrshort{hdl} particles \cite{Nguyen2010MolecularE4}. The lower affinity of ApoE4 for HDL particles leads to increased levels of unlipidated ApoE, resulting in its aggregation \cite{Hatters2006ApolipoproteinFunction}. Moreover, ApoE4 fibrils are more neurotoxic than those of ApoE2 and ApoE3 \cite{Hatters2006Amino-terminalFibrils}.

Poor lipidation leads to poor ApoE recycling \cite{Yassine2020APOEDisease} (Fig. \ref{ApoeEffectsA}). The latter favors the entrapment of ABCA1 in endosomes, away from the cell membrane, thereby pooling cholesterol in the cell membrane rather than attaching it to HDL particles \cite{Rawat2019ApoE4Astrocytes}. This increased cholesterol content in the cell membrane amplifies \acrfull{tlr4} signaling in macrophages, activating \acrshort{nfkb} and inducing an inflammatory gene response \cite{Yassine2020APOEDisease}.  ApoE4 accumulation also sequestrates the insulin receptor (IR) in endosomes, impacting cellular energy preferences \cite{Zhao2017ApolipoproteinEndosomes}. This leads to a decrease in glucose utilization for \acrshort{atp} production (Fig. \ref{ApoeEffectsB}) and an increase in fatty acid oxidation \cite{Svennerholm1997ChangesSwedes}. 

\begin{figure}[t]
  \includegraphics[width=0.8\textwidth]{figures/ApoEeffectsB.jpg}
    \caption{Effects of ApoE isoforms on the metabolism and removal of A$\beta$. A$\beta$ is primarily cleaved from \acrfull{app}. In the brain, ApoE is mainly expressed in astrocytes and microglia, and undergoes lipidation by \acrfull{abca1} to create lipoprotein particles. ApoE increases the accumulation and buildup of A$\beta$, or promotes cellular uptake of A$\beta$ by astrocytes or microglia via endocytosis of the lipidated ApoE-A$\beta$ in an isoform-specific manner. This process involves several receptors, such as \acrfull{ldlr} and \acrfull{lrp}. ApoE also facilitates isoform-specific breakdown of A$\beta$ outside the cells. At the \acrlong{bbb}, soluble A$\beta$ is predominantly transported from the \acrfull{isf} into the bloodstream via LRP1 and P-glycoproteins. Additionally, ApoE plays a role in the perivascular drainage of A$\beta$. Insufficient clearance of A$\beta$ can lead to its accumulation in the brain tissue, contributing to the formation of A$\beta$ oligomers and amyloid plaques. Source: \citetitle{Husain2021APOETherapeutics}, \Citeauthor{Husain2021APOETherapeutics} (\citeyear{Husain2021APOETherapeutics}) \cite{Husain2021APOETherapeutics}}
  \label{ApoeEffectsA}
\end{figure}

\begin{figure}[b]
  \includegraphics[width=0.8\textwidth]{figures/ApoEeffectsA.jpg}
    \caption{Schematic overview of A$\beta$-independent effects of ApoE in AD pathology. ApoE4 increases the phosphorylation of tau proteins, leading to the creation of tangles, inducing neurodegeneration, synaptic deficits and neurotoxicity. Moreover, ApoE4 is associated with decreased cholesterol efflux and neuroinflammation mediated by \acrfull{il}, \acrfull{tnf}, \acrfull{ros}. In the mitochondria, ApoE4 impairs glucose metabolism and ATP production.  Source: \citetitle{Husain2021APOETherapeutics}, \Citeauthor{Husain2021APOETherapeutics} (\citeyear{Husain2021APOETherapeutics}) \cite{Husain2021APOETherapeutics}}
  \label{ApoeEffectsB}
\end{figure}

\subsubsection{Interplay between ApoE lipidation and Alzheimer's disease pathophysiology}\label{ApoEAD}
As mentioned earlier, ApoE isoforms have differential pleiotropic effects on various cellular functions. ApoE4 induces a pro-inflammatory response, leading to the dysfunction of the \acrlong{bbb}, which in turn impairs cognitive functions \cite{Marottoli2017PeripheralDysfunction, Teng2017ApoEInjury, Kloske2020TheDisease}. Moreover, ApoE modulates the primary neuropathological hallmarks of \acrshort{ad}: neuroinflammation, A$\beta$ plaques and tau tangles \cite{Husain2021APOETherapeutics}. Evidence from human and transgenic mice studies reveals increased brain A$\beta$ and amyloid plaque loads in ApoE4 carriers, compared to ApoE3; with the lowest levels in ApoE2 carriers \cite{Huang2017ApoE2Secretion, Tachibana2016RescuingLRP1, Safieh2019ApoE4:Disease}. ApoE4 has a higher binding affinity for A$\beta$ which leads to impaired clearance, its intracellular aggregation and higher plaque loads \cite{Kloske2020TheDisease} (Fig. \ref{ApoeEffectsA}). Additionally, high levels of ApoE4 in neurons remarkably increase tau protein phosphorylation, while high concentrations of ApoE3 do not seem to have an effect \cite{Cao2017ApoE4-associatedInjury, Shi2017ApoE4Tauopathy, Vasilevskaya2020InteractionAthletes, Wang2018GainCorrector}. Notably,  ApoE directly inhibits phosphorylation of tau by \acrfull{gsk} \cite{Hoe2006ApolipoproteinNeurons}. An overview of the A$\beta$-(in)dependent effects of ApoE is shown in Fig. \ref{ApoeEffectsB}.

\subsection{ApoE4-mediated metabolic changes in Alzheimer's disease}
Metabolism entails the repertoire of chemical reactions that keep living organisms alive. Metabolites  --especially lipid \cite{Barupal2019SetsPathophysiology,Fernandez-Calle2022APOEDiseases, Proitsi2017AssociationAnalysis}--, perceived as functional intermediates of \acrshort{ad} development, are rigorously studied for biomarkers or targets for treatment \cite{Oeckl2019GlialImpairment}.
 
\subsubsection{Measured in \textit{post-mortem} brain tissue} A metabolomic profiling of brain tissue, obtained \textit{post-mortem} from \acrshort{ad} patients and healthy controls showed  pronounced impairments in sterol and sphingolipid levels in ApoE4 carriers with \acrshort{ad}  \cite{Bandaru2009ApoE4Brain}. However, another \textit{post-mortem} metabolomic analysis did not reveal nuances significantly correlated with ApoE4 \cite{Novotny2023MetabolomicBrains}, although they showed trends in increased cholesterol esters, unsaturated lipids, and sphingomyelin species.

\subsubsection{Measured in blood}
Transcriptomic and lipidomic analyses in humanized ApoE mice associated ApoE4 with decreased free fatty acid levels, many increased  tricarboxylic acid (TCA) cycle metabolites, as well as changes in plasma levels of phosphatidylcholines and unsaturated fatty acids \cite{Area-Gomez2020APOE4Mice, Zhao2020AlzheimersPathways}. A recent study on 58 individuals found six downregulated plasma metabolites (including lysophospholipids and cardiolipin) in ApoE4 carriers \cite{pena-bautista2020MetabolomicsEffect}. Further, the plasma metabolome of the latter reveals a preference for aerobic glycolysis \cite{Farmer2021APO4Glycolysis}. Significant correlations of ApoE genotype and sex with metabolites were observed, i.e. several phosphatidylcholines were found in a large study of more than 1500 individuals \cite{Arnold2020SexMetabolome}.

Perturbed serum metabolites associated with \acrshort{ad} are aminoacids, amines \cite{deLeeuw2017Blood-basedDisease, Green2023InvestigatingDisease}, cholesteryl esters \cite{Proitsi2017AssociationAnalysis}, sphingolipids \cite{Varma2018BrainStudy,Sun2022AssociationDisease,Green2023InvestigatingDisease,Oeckl2019GlialImpairment,Barupal2019SetsPathophysiology}, fatty acids \cite{Fernandez-Calle2022APOEDiseases,deLeeuw2017Blood-basedDisease}, glycerophospholipids \cite{Varma2018BrainStudy, Jia2022ATypes,Huo2020BrainAnalysis, Weng2019TheImpairment}, phosphatidylcholines \cite{Simpson2016BloodAging} and lipid peroxidation compounds \cite{Fernandez-Calle2022APOEDiseases}. These molecules are usually identified via high-throughput metabolomic pipelines (coupled with Mass Spectrometry (MS) detectors) that trace all compounds in a sample and result in high-dimensional data \cite{Oka2023MultiomicsCohort}. The latter often require advanced statistical methods e.g. projection to latent structures \cite{Weng2019TheImpairment, Peeters2019StableData} or graphical models \cite{Peeters2022Rags2ridges:Matrices} in order to extract putatively meaningful information. 
With such techniques, de Leeuw et al. discovered distinct serum metabolic signatures among \acrshort{ad} patients-controls and those carrying at least one ApoE4 allele \cite{deLeeuw2017Blood-basedDisease}, as they are shown in Fig. \ref{netan17}. The different metabolic profiles, however, among ApoE4 non-carriers remain obscure. The present data science approach shows ApoE4-mediated differentially expressed metabolites, potentially unveiling distinct pathways of metabolic deregulation in \acrshort{ad}.

\begin{figure}[htb]
\vspace*{-0.2cm}
  \includegraphics[width=0.9\textwidth]{figures/network.jpeg}
    \caption{Mutual (left-hand panel) and distinct (right-hand panel) metabolic network topologies between ApoE4 carriers with AD and non-carriers with SCD, as published by de Leeuw et al.. Red edges represent links that are present exclusively in ApoE4 carriers with AD. Green edges represent connections found in the SCD group without ApoE4. Solid edges represent positive partial correlations, while dashed edges represent negative partial correlations. Abbreviations: SCD, subjective cognitive decline. Source: \citetitle{deLeeuw2017Blood-basedDisease}, \citeauthor{deLeeuw2017Blood-basedDisease} (\citeyear{deLeeuw2017Blood-basedDisease}) \cite{deLeeuw2017Blood-basedDisease}}
  \label{netan17}
\end{figure}

\newpage
\subsection{Research Questions}
ApoE4 carriers --particularly females-- experience metabolic disturbances and are at increased risk of \acrshort{load}. The mechanistic links, however, between ApoE4 dose, metabolism and \acrshort{ad} development are not entirely known \cite{Fernandez-Calle2022APOEDiseases}. Tracking ApoE4 dose effects on serum metabolites might reveal metabolic perturbations at systemic level, preceding or concurring with AD. Hence, in an effort to elucidate ApoE4-mediated changes in serum metabolites in AD and SCD, one could state the following research questions (RQ):

Are there mechanistic links between ApoE4 dose and serum metabolome in AD?
\begin{enumerate}
    \item Are there ApoE4 dose effects on serum metabolite levels in AD?
    \item What is the (added) classification potential of serum metabolites in predicting ApoE4 and AD status?   
    \item How do the covariance network topologies of metabolites differ between ApoE4 carriers and non-carriers?
\end{enumerate}

\subsection{Approach and Overview}
An introduction to the data used in this study is found in Section \ref{subjects}. A general overview of the software is found in Section \ref{datamanagement}, while the R session information is found in Appendix \ref{appendixB}. To facilitate statistical analysis, two new features were created for the first two research questions: "ApoE4Dose" (0, 1, 2) and "ApoE4AD" (4 possible phenotypes: AD without ApoE4, AD with at least 1 ApoE4, SCD without ApoE4, SCD with at least 1 ApoE4), respectively, as described in Section \ref{featureeng}.

The statistical methods applied to answer the research questions were adapted from \citeauthor{deLeeuw2017Blood-basedDisease}'s \citetitle{deLeeuw2017Blood-basedDisease} and are described in Section \ref{stats}. To screen for ApoE4 dose effects on serum metabolites, two approaches were taken: a global test (correcting only for sex) and nested linear model comparison using ANCOVA F-tests (correcting for several factors: Table \ref{tab:clin}, except CSF markers). 

To test the (added) classification potential of serum metabolites against ApoE4AD, several multi-class classification models were fitted. First, a benchmark multinomial logistic regression model was fitted using only clinical background data as predictors. Second, the full metabolomic panel was addded on top of the clinical data in the same model and feature importance was calculated. Third, the metabolites were projected to a latent orthogonal space, where 6 meta-metabolites (accounting for around 30\% of the variance) replaced the original metabolites. Finally, the meta-metabolites were fitted on top of the clinical data in a multinomial logistic regression model, a decision tree and an \acrfull{xgb} model. The discriminatory performance of the aforementioned models was then comprehensively evaluated and compared.

To visualise and compare the covariance network topologies of metabolites among ApoE4 carriers and non-carriers, the high precision matrices were first regularised with Ridge, sparsified and then prunned controlling the False Discrovery Rate. Network topologies were plotted and their statistics were calculated and compared between the ApoE4 carriers and non-carriers using Wilcoxon Signed Rank test.

The results of the analysis are presented in Section \ref{results} and discussed in Section \ref{discuss}. The study is concluded in Section \ref{concl}


\newpage
\section{Methods}\label{methods}

\subsection{Subjects}\label{subjects}
The data were collected from 120 AD patients and 127 SCD ($n$ = 247 in total) individuals recruited within Amsterdam Dementia Cohort \cite{VanDerFlier2018AmsterdamCare, deLeeuw2017Blood-basedDisease}. In this study two data sets were used: a targeted metabolomics panel and clinical background data, i.e. age at diagnosis, sex, smoking status, alcohol consumption, hypertension (and medication), hyperlipidemia (and medication), anticoagulant medication, antidepressants, mean arterial pressure (MAP) and body mass index (BMI) (see Table \ref{tab:clin}). The metabolomics set contains $p =$ 230 metabolites (amines, organic acids, lipids and oxidative stress compounds). The methodology for the metabolomic analysis and ApoE genotyping can be found at de Leeuw et al.'s  Blood-based metabolic signatures in Alzheimer's disease \cite{deLeeuw2017Blood-basedDisease}: SMT1. The data were cleaned as described in the same article. The resulting datasets for AD and SCD are high-dimensional, in the sense that they contain more variables than observations ($p > n$). Another particularity of the data is the covariance and collinearity of the variables. Therefore, appropriate measures need to be taken to prevent model over-fitting --the algorithm being unable to distinguish signal from noise and fitting the latter--  and to correct for spurious correlations.

\begin{table}[htb]
\caption{Clinical background characteristics used as control variables of ApoE4 (dose) effects.}
\label{tab:clin}
\begin{threeparttable}
\begin{tabular}{clllll} \toprule
                                & \textbf{Clinical feature}   & \textbf{Type} \\ \midrule
\multirow{2}{*}{Anthropometric} & Age                         & Discrete      \\
                                & Sex                         & Binary        \\
\multirow{2}{*}{Intoxications}  & Smoking (past, current, no) & Nominal       \\
                                & Alcohol                     & Binary        \\
\multirow{3}{*}{Comorbidities}  & Hypertension                & Binary        \\
                                & Diabetes mellitus           & Binary        \\
                                & Hypercholesterolemia        & Binary        \\
\multirow{3}{*}{Medication}     & Cholesterol lowering        & Binary       \\
                                & Antidepressants             & Binary        \\
                                & Antiplatelet                & Binary        \\ 
\multirow{3}{*}{CSF markers$\ast$}    & A$\beta_{42}$               & Continuous       \\
                                & tau                         & Continuous       \\
                                & p-tau                       & Continuous       \\\bottomrule
\end{tabular}
\begin{tablenotes}
  \item[$\ast$] were only used in the ApoE4AD status multi-class classification
\end{tablenotes}
\end{threeparttable}
\end{table}

\subsection{Data management}\label{datamanagement}
The FAIR principles for data management and stewardship in science were published by  ~\citeauthor{Wilkinson2016TheStewardship} in 2016 \cite{Wilkinson2016TheStewardship}. FAIR stands for Findable, Accessible, Interactive, and Reusable data; the intention is to create and use data that are well-documented and reproducible. These principles were considered at every step of the study and implemented when applicable. The statistical analysis was performed in R (version 4.3.2), and the current report was written in \LaTeX  (Tex Live version 2023). All files are stored in a private Github repository --with git as \acrfull{vcs}.

\subsection{Feature Engineering}\label{featureeng}
The information of the ApoE genotypes is valuable, and it might be interesting to screen for metabolic nuances between them. However, the genotypes were not equally represented in the data and hence, testing for differences among them would not be possible. The following two subsections describe the features ApoE4Dose and ApoE4AD that were created to balance the genotypes.

\begin{table}
  \caption{Observed ApoE genotype (top part), allele (middle part) and ApoE4 allelic (bottom part) counts in AD and SCD}
\begin{threeparttable}
\centering 
\label{tab:ApoEfreq}
  \begin{tabular}{crrrr} \toprule
    \multicolumn{1}{l}{}                & \multicolumn{1}{l}{}               & AD (\%)    & SCD (\%)   & Total (\%)  \\ \midrule
    \multirow{6}{*}{\textit{genotypes}} & $\varepsilon_2/\varepsilon_2$        & 2 (1.7)    & 0 (0.0)    & 2 (0.8)     \\
                                        & $\varepsilon_2/\varepsilon_3$      & 15 (12.5)  & 3 (2.4)    & 18 (7.3)    \\
                                        & $\varepsilon_2/\varepsilon_4$      & 5 (4.2)    & 2 (1.6)    & 7 (2.8)     \\
                                        & $\varepsilon_3/\varepsilon_3$      & 69 (57.5)  & 37 (29.1)  & 106 (42.9)  \\
                                        & $\varepsilon_3/\varepsilon_4$      & 26 (21.7)  & 59 (46.5)  & 85 (34.4)   \\
                                        & $\varepsilon_4/\varepsilon_4$      & 3 (2.5)    & 26 (20.5)  & 29 (11.7)   \\ \midrule
   \multirow{3}{*}{\textit{alleles}}    & $\varepsilon_2$    & 24 (10.0)  & 5 (2.0)    & 29 (6.0)    \\
                                        & $\varepsilon_3$    & 179 (74.6) & 136 (53.5) & 315 (64.0)    \\
                                        & $\varepsilon_4$   & 37 (15.4)  & 113 (44.5) & 150 (30.0)    \\ \midrule
    \multirow{4}{*}{$\varepsilon_4$}   & 1x$^\ast$            & 31 (25.8)  & 61 (48.0)  & 92 (37.2)   \\
                                        & 2x$^\ast$            & 3 (2.5)    & 26 (20.5)  & 29 (11.7)   \\
                                        & $\geq$ 1x$ ^\dagger$ & 34 (28.3)  & 87 (68.5)  & 121 (49.0)  \\
                                        & No$^{\ast\dagger}$     & 86 (71.7)  & 40 (31.5)  & 126 (51.0)  \\ \midrule
    \multicolumn{2}{r}{Total}                                                & 120 (49.0)       & 127 (51.0)       & 247         \\ \bottomrule
  \end{tabular}
  \begin{tablenotes}
    \item[$\ast$] rows used for ApoE4Dose
    \item[$\dagger$] rows used for ApoE4AD 
  \end{tablenotes}
\end{threeparttable}
\end{table}

\subsubsection{ApoE4 dose effects}
\citeauthor{deLeeuw2017Blood-basedDisease} divided the subjects into two classes: those carrying at least one $\varepsilon_4$ allele, and $\varepsilon_4$ non-carriers. However, in order to study the $\varepsilon_4$ \textit{dose} effects, the genotypes can be categorized into groups, based on the number of $\varepsilon_4$ alleles they carry: 0, 1 or 2. The number of ApoE4 allele doses is shown in the bottom part of Table \ref{tab:ApoEfreq}.

\subsubsection{Classification of ApoE4 and AD status}
In order to incorporate ApoE4 status, as well as the diagnosis of AD, a four-level feature (AD without ApoE4, AD with at least 1 ApoE4, SCD without ApoE4, SCD with at least 1 ApoE4, "ApoE4AD") was created, as shown in \ref{tab:ApoEfreq}.

\subsection{Statistical Analysis} \label{stats}
\subsubsection{ApoE4 dose effects on serum metabolite levels in AD} \label{rq1}
To test if the number of ApoE4 alleles have an effect on mean metabolite levels in \acrshort{ad}, two methods were applied: a global test and nested linear model comparison with ANCOVA.

\leavevmode\newline\textbf{Global Test}\hspace{.25cm} The concept of a global test was first introduced by \citeauthor{Simon2004DesignHealth}, and proposed an approach based on permutations to cater to the high dimensionality of gene expression data. The R package \textsf{globaltest} \cite{Goeman2004AOutcome, Goeman2006TestingAlternative, Goeman2023ThePackage} developed by \citeauthor{Goeman2004AOutcome} features a multinomial logistic regression model, fitting genes to predict clinical or biological group membership (number of ApoE4 alleles in this case). This method is also appropriate for other types of -omics data, such as metabolomics in this study \cite{Goeman2023ThePackage}. The null hypothesis is that metabolite levels are independent of the ApoE4 dose X, i.e. $H_0 : P(Y|X) = P(Y)$, where $X \in \mathbb{R}^{n x p}$. The test statistic under $H_0$ follows, asymptotically, a normal distribution. The \texttt{gt} function of the package was used to screen for nuances in metabolite levels on the number of ApoE4 alleles, correcting for sex. To assess ApoE4 dose effects correcting for clinical data nested linear model comparison was performed, as described in the next section.

\leavevmode\newline \textbf{Nested linear models}\hspace{.25cm}
An linear model with the Least Squares estimation creates a regression using categorical and numerical variables, while minimizing the Residual Sum of Squares (RSS) \cite{ott2015introduction}, 
$$\mathrm{RSS} = \sum (y - \hat{y})^2$$ where $y$ is the response and $\hat{y}$ is its regression estimate. Two models were fitted for each metabolite, a full and a nested model. The dependent variable in each model was a metabolite; the nested model \eqref{rm}, has only clinical variables (Table \ref{tab:clin} except CSF markers) while the full model \eqref{fm}, features the clinical variables, plus the number of ApoE4 alleles (0, 1 or 2 -nominal) as explanatory variables. Let $y_j$ represent the $j$-th metabolite, $x_k$ the $k$-th clinical variable, and $D_{\varepsilon_4}$ the ApoE4 dose; the nested models then are:
\begin{align}
    & y_j = \beta_0 + \sum_{k=1}^m\beta_kx_k +\epsilon \label{rm} \\
    & y_j = \beta_0 + \sum_{k=1}^m\beta_kx_k + \beta_{m+1}D_{\varepsilon_4} + \epsilon \label{fm}
\end{align}
For every metabolite $j$ = 1,...,$p$ the hypothesis test is 
\[H_0: \beta_{m+1} = 0 \; \mathrm{vs} \; H_\alpha: \beta_{m+1} \neq 0\]
The test statistic is ANCOVA F-test:
\[ F = \frac{\mathrm{SSReg}_{full}-\mathrm{SSReg}_{nested}/df_{full}-df_{nested}}{\mathrm{RSS}_{nested}/n-(m+2)}\]
Under H$_0$ $F$ follows an $F_{1, n-(m+2)}$ distribution \cite{ott2015introduction}. The null hypothesis is rejected for large values of $F$.

This implies $p$ hypothesis tests, which qualifies as multiple testing (the probability of incorrectly rejecting H$_0$, type I error, increases monotonically with every additional test). A method to treat the latter is controlling False Discovery Rate \cite{Benjamini1995ControllingTesting}, that is controlling the expected ratio of incorrectly rejected H$_0$ hypotheses, globally e.g. at an $\alpha$ of 0.05. With the Benjamini-Hochberg's approach, first, the p-values are sorted in ascending order. Then, for every p-value $j$, $\alpha$ is multiplied by $j$ over the total number of tests \cite{Benjamini1995ControllingTesting} $m=230$ in the AD group in this study. In other words, after adjustment, a null hypothesis $j$ may be rejected only if its associated F-test's p-value is less or equal to a fraction ($j/m$) of $\alpha$.

\begin{algorithm}
\caption{Benjamini–Hochberg's procedure to control FDR}\label{alg:fdr}
Specify $\alpha$, the level at which to control the FDR.\\
Compute p-values, $p_1, ... , p_m$, for the $m$ null hypotheses $H_{01},...,H_{0m}$. \\
Order the $m$ p-values so that $p(1) \leq p(2) \leq ... \leq p(m)$.\\
Define
\[L = \max\{j : p(j) \leq \frac{j}{m}\alpha\}\] \\
Reject all null hypotheses $H_{0j}$ for which $p_j \leq p(L)$.
\end{algorithm}

To test for ApoE4 dose effects on each metabolite in AD and SCD, $2m=460$ model comparisons were needed in total, hence a function was created to iterate over the metabolites, fit the nested models, compare them using ANCOVA F-tests, store and adjust the p-values, filter those below .05, then consolidate the coefficients of the meaningful full models and the p-values of their t-tests in a table and display it. To decrease run time, as well as harness the power of multiple CPU cores, \textsf{furrr}'s \texttt{future\_map} function was used for parallel iterations.

\subsubsection{Classification of ApoE4 and AD status}\label{rq2}
In machine learning, a class denotes a group of objects that share common characteristics, such as having \acrshort{ad} or ApoE4 \cite*{Drummond2010}. Classification, in this context, denotes training a (supervised) learning algorithm on labelled data (containing their class) \cite*{Drummond2010}. The classifier learns patterns in the data and is then able to predict class membership for unknown data \cite*{Drummond2010}.

The R package \textsf{caret} streamlines the training and comparison of classification and regression models, offering broad model training options and feature importance estimation \cite{Kuhn2008BuildingPackage}. The functions \texttt{trainControl} and \texttt{train} were used to fit the models. A sampling method used to deal with the unbalanced classes was \acrshort{smote} (Synthetic Minority Over-Sampling Technique), as implemented by the package \textsf{DMwR2} \cite{DMwR2}.

\leavevmode\newline \textbf{Bias-Variance trade-off}\hspace{.25cm}The degree to which a user can understand and interpret the prediction or decisions made by a statistical model is defined as \textit{interpretability} \cite{Elshawi2019OnHypertension}. It was of interest in this study to find the optimal balance between the performance of a model and its interpretability. The \textit{bias-variance trade-off} was formally introduced by \citeauthor{Geman1992NeuralDilemma} and refers to the trade-off between the accuracy (opposite of bias) and precision (opposite of variance) of a prediction. It also refers to the trade-off between model flexibility (or complexity) and interpretability \cite{Geman1992NeuralDilemma}. One may consider this trade-off during model and evaluation method selection, as some impose more bias or variance than others.

\leavevmode\newline \textbf{Multi-class classification models}\hspace{.25cm}Considering interpretability, Multinomial Logistic Regression (\acrshort{mnl}) is inherently interpretable. Let $y$ a response with $K$ classes, $k \in N, [1,4]$ representing the k-th class of ApoE4AD and $\beta_{kj}$ its set of coefficients,  $\beta_{lj}$ the coefficients of the rest of classes for $j$-th metabolite, then a \acrlong{mnl} model calculated the probability

\[\textrm{P}(y=k|X=x) =  \dfrac{e^{\sum_{j=1}^{p}\beta_{kj}x_j}}{\sum_{l=1}^{K}\sum_{j=1}^{p}e^{\beta_{lj}x_j}}\]

When $p > n$, the coefficient estimation method has low bias and high variance, in that small changes in the training data can result in very different coefficient estimates \cite{James2023AnEdition}. 

Regularization trades off a small increase in bias for a great decrease in variance, by shrinking the low coefficients towards zero \cite{James2023AnEdition}. Ridge or L2 regularization \cite{Cessie1992RidgeRegression} shrinks the coefficients without setting them to exactly zero \cite{Cessie1992RidgeRegression}
\[\textrm{RSS} + \lambda\sum_{j=1}^{p}\beta_j^2 \]where $\lambda \geq 0$ is a tuning parameter that balances the coefficient shrinking effect. The function \texttt{multinom} of the package \textsf{nnet} fits a shallow neural network (with a single hidden layer, but allowing skip-layer connections) \cite{nnet}. For the $i$-th observation, it calculates the weights using a Least Squares estimation of the negative conditional log-likelihood that it belongs to the $k$-th class
\[E = \sum_{i}\sum_{k}-y_{ik}\log\hat{y}_{ik}, \; \;  \hat{y}_{ik} =  \dfrac{e^{\sum_{j=1}^{p}\beta_{kj}x_{ij}}}{\sum_{l=1}^{K}\sum_{j=1}^{p}e^{\beta_{lj}x_{ij}}}\]
where $y_{ik}$, the true class will be exactly one and the others all zero \cite{nnet}. It regularises the fit via weight decay, a Ridge-like penalty that uses the sum of squares of the weights \cite{nnet}.

Another method to treat multi-collinearity and high dimensionality is a 2-stage \acrfull{ml} Factor Projection to a Latent Orthogonal space, such as the one the package \textsf{FMradio} \cite{Peeters2019StableData} performs. In the 1st stage, a L2-regularised ML estimation is used to filter out redundant features from the data matrix. In the second stage, the aforementioned matrix is projected to an orthogonal space where the features are replaced by -fewer- factors that explain their covariance. One can then use the produced factor scores as predictors in any model.

Decision Trees (\acrshort{dt}) are inherently interpretable, non-parametric models, that fit well large and complicated data sets. They have a tree-like structure that splits the data based on a threshold into branches and leaves (nodes) \cite{Song2015DecisionPrediction}. The \texttt{rpart2} \cite{rpart} function of \textsf{rpart} was used.

Boosting models, are ensemble models that fit several weak learners (such as linear/logistic regression or DTs) sequentially, reweighing the data, and take their weighted majority vote \cite{Friedman2000boosting,Friedman2001gbm}. Despite Boosting tends to outperform DTs, it often operates as a \textit{black box} and is poorly interpretable. The state-of-the-art \acrlong{xgb} (\texttt{xgbTree}) of the package \textsf{xgboost} \cite{Chen2016XGBoost:System} was used.

\begin{algorithm}
\caption{Multi-class classification of ApoE4 and AD status pipeline} \label{alg:classification}
    Fit clinical data only in \acrlong{mnl}\\
    Fit clinical data + 230 serum metabolites in \acrlong{mnl}\\
    Fit clinical data + 6 meta-metabolites in \acrlong{mnl}\\
    Fit clinical data + 6 meta-metabolites in a Decision Tree\\
    Fit clinical data + 6 meta-metabolites in an XGBoost\\
    Evaluate performance and compare
\end{algorithm}

\leavevmode\newline \textbf{Implementation \& Evaluation}\hspace{.25cm} First, a benchmark model was created fitting the clinical background data to predict ApoE4AD in a penalised \acrlong{mnl} model. Second, the 230-metabolite panel was fitted on top of the clinical background data in a penalised \acrlong{mnl} model ($\lambda$ = 9.187724, obtained from repeated 10-fold Cross-Validation (CV)) and feature importance was calculated. Then the full metabolite panel was projected into 6 latent factors (meta-metabolites), explaining around 30\% of their variance. The 6-factor metabolite projection was then fitted on top of the clinical data in a penalised \acrlong{mnl} ($\lambda$ = 0), a \acrlong{dt} and an XGBoost model. The aforementioned models were hyperparameter tuned over a grid of values and the best was selected using repeated (100 times) 10-fold \acrshort{cv}. 

Model performance was comprehensively assessed and compared, with the repeated 10-fold CV-obtained \acrfull{roc} curves and their respective \acrfull{auc} using the \textsf{pROC} package \cite{pROC} and other metrics such as accuracy, precision, recall and F1-score from \textsf{caret}'s \texttt{confusionMatrix}. Feature importance was estimated using \textsf{caret}'s \texttt{varImp}.

\subsubsection{Metabolite Covariance Network Analysis}\label{rq3}
Network science offers a unifying framework for data and system representation, applicable to any domain \cite{Barabasi2015NetworkScience}. A network, in an abstract sense, consists of nodes connected with links, also referred to as edges. In data science, a network whose nodes represent random features, whose joint probability distribution is defined by the ensemble of their edges is called \textit{graphical model} \cite{Peeters2022Rags2ridges:Matrices}. A metabolomic covariance correlation network represents the ensemble of metabolites based on their covariance, showing nuances among the samples that non-graphical statistical methods on individual metabolites may fail to detect \cite{PerezDeSouza2020Network-basedInterpretation}. It may provide insights into correlated metabolites that do not belong in the same metabolic pathway \cite{PerezDeSouza2020Network-basedInterpretation}.


A \textit{Gaussian graphical model} (\acrshort{ggm}) is an undirected graph that represents the conditional independence properties of the features \cite{KollerProbabilisticTechniques}. The statistic employed by GGMs is the partial correlation which also adjusts for indirect correlation, i.e. two metabolites are correlated with a third one and are shown correlated with each other \cite{Amara2022NetworksInterpretation}. For instance, let $\mathcal{G=(V,E)}$ be a GGM consisting of a set $\mathcal{V}$ of $p$ vertices, corresponding to random features $Y_1,...,Y_p$ with joint probability distribution $P \sim N_p(\mathbf{0, \Sigma})$, and set of edges $\mathcal{E}$, such that for all pairs $\{Y_i , Y_j\}$ with $i\neq j$:

\[ \mathbf{\Sigma}_{ij}^{-1} = (\mathbf{\Omega}_{ij})=0 \Longleftrightarrow Y_i \Perp Y_j\mid\{Y_k : k \neq i,j\} \Longleftrightarrow (i, j) \notin \mathcal{E} \]

In natural language, a zero value in the inverse covariance matrix (usually referred to as precision matrix $\mathbf{\Omega}$) implies that the respective random features are independent, given the rest of features, and they are not connected by an undirected edge $((i, j) \notin \mathcal{E})$ \cite{Peeters2022Rags2ridges:Matrices}. In this study, the package \textsf{rags2ridges} \cite{Peeters2022Rags2ridges:Matrices} was used to generate the feature covariance matrices of ApoE4 carriers and non-carriers with AD, compute their precision matrices, regularise them and represent them in GGMs. Communities of coregulated metabolites were identified using the community search algorithm by Girvan-Newman \cite{PhysRevE.69.026113} as implemented in \textsf{rags2ridges}.

\clearpage
\section{Results} \label{results}
The ApoE genotype frequencies in \acrshort{ad} and SCD are shown in Table \ref{tab:ApoEfreq}. Cumulatively, the most abundant allele is $\varepsilon_3$ (64\%), followed by $\varepsilon_4$ (30\%) and $\varepsilon_2$ (6\%). In AD, the allelic frequencies were 74.6\% for $\varepsilon_3$, followed by 15.4\% for $\varepsilon_4$ and 10\% for $\varepsilon_2$. In the SCD group, $\varepsilon_3$, $\varepsilon_4$ and $\varepsilon_2$ showed relative frequencies of 53.5\%, 44.5\% and 2\%. The most common genotype was $\varepsilon_3/\varepsilon_3$, followed by $\varepsilon_3/\varepsilon_4$ and $\varepsilon_4/\varepsilon_4$.
\subsection{ApoE4 dose effects on serum metabolite levels in AD}
\subsubsection{Global Test}
Testing for ApoE4 dose-effect on serum metabolites of AD patients, correcting for sex (Ho: ApoE4 dose has no effect on mean metabolite levels, Ha: it has an effect), showed a significant global difference in metabolites (p = 0.017). The most significantly affected metabolites are triglycerides and diglycerides (FDR-adjusted p-value $<$0.05) See Table \ref{tab:gt} and Fig. \ref{plot:gt}. In SCD individuals, the global test showed no significant differences among ApoE4 doses (p=0.544). Generally, fewer metabolites were altered in this group and none of the associations remained significant after adjusting for FDR. Interestingly, the affected metabolites are different between AD and SCD.

\begin{table} 
\vspace{-0.5cm}
\makebox[\linewidth]{\begin{threeparttable}
\caption{Metabolites affected by ApoE4 dose, correcting for sex as per globaltest \cite{Goeman2023ThePackage}}.
\label{tab:gt}
\begin{tabular}{cclccrr} \toprule
  & Class & Metabolite & \multicolumn{1}{l}{Inheritance} & Assoc. with & \multicolumn{1}{l}{p-value} & \multicolumn{1}{l}{FDR} \\ \midrule
  \multirow{39}{*}{AD} & Lipid & TG (56:2) & 0.046 & 1 ApoE4 & $<$0.001 & 0.016 \\
   & Lipid & TG (58:1) & 0.117 & 1 ApoE4 & $<$0.001 & 0.021 \\
   & Lipid & DG (36:3) & 0.19 & 1 ApoE4 & $<$0.001 & 0.021 \\
   & Lipid & TG (56:3) & 0.244 & 1 ApoE4 & $<$0.001 & 0.021 \\
   & Lipid & TG (52:3) & 0.424 & 1 ApoE4 & 0.001 & 0.031 \\
   & Lipid & TG (54:5) & 0.48 & 1 ApoE4 & 0.001 & 0.027 \\
   & Lipid & TG (58:2) & 0.722 & 1 ApoE4 & 0.001 & 0.027 \\
   & Lipid & TG (54:4) & 0.761 & 1 ApoE4 & 0.002 & 0.033 \\
   & Lipid & TG (58:9) & 1 & 1 ApoE4 & 0.001 & 0.027 \\
   & Lipid & TG (56:7) & 1 & 1 ApoE4 & 0.001 & 0.027 \\
   & Lipid & TG (58:8) & 1 & 1 ApoE4 & 0.001 & 0.032 \\
   & Lipid & TG (54:6) & 1 & 1 ApoE4 & 0.002 & 0.035 \\
   & Lipid & TG (56:8) & 1 & 1 ApoE4 & 0.002 & 0.037 \\
   & Lipid & TG (52:4) & 1 & 1 ApoE4 & 0.003 & 0.055 \\
   & Lipid & TG (54:3) & 1 & 1 ApoE4 & 0.004 & 0.055 \\
   & Lipid & TG (56:6) & 1 & 1 ApoE4 & 0.004 & 0.059 \\
   & Lipid & TG (56:1) & 1 & 1 ApoE4 & 0.004 & 0.059 \\
   & Lipid & TG (52:2) & 1 & 1 ApoE4 & 0.005 & 0.063 \\
   & Lipid & TG (54:2) & 1 & 1 ApoE4 & 0.005 & 0.063 \\
   & Lipid & TG (60:2) & 1 & 1 ApoE4 & 0.007 & 0.077 \\
   & Lipid & TG (58:10) & 1 & 1 ApoE4 & 0.011 & 0.124 \\
   & Lipid & TG (51:3) & 1 & 1 ApoE4 & 0.015 & 0.154 \\
   & Lipid & SM (d18:1/18:1) & 1 & 2 ApoE4 & 0.018 & 0.169 \\
   & Organic acid & 2-ketoglutaric.acid & 1 & 2 ApoE4 & 0.019 & 0.169 \\
   & Lipid & TG (52:0) & 1 & 1 ApoE4 & 0.019 & 0.169 \\
   & Lipid & TG (50:3) & 1 & 2 ApoE4 & 0.02 & 0.169 \\
   & Organic acid & Uracil & 1 & no ApoE4 & 0.02 & 0.169 \\
   & Lipid & TG (52:5) & 1 & 1 ApoE4 & 0.021 & 0.174 \\
   & Lipid & TG (54:0) & 1 & 1 ApoE4 & 0.022 & 0.174 \\
   & Lipid & TG (50:2) & 1 & 2 ApoE4 & 0.022 & 0.174 \\
   & Lipid & TG (51:2) & 1 & 1 ApoE4 & 0.023 & 0.174 \\
   & Aminoacid & L-Glutamine & 1 & no ApoE4 & 0.024 & 0.174 \\
   & Lipid & TG (50:1) & 1 & 2 ApoE4 & 0.025 & 0.174 \\
   & Lipid & TG (50:4) & 1 & 1 ApoE4 & 0.026 & 0.176 \\
   & Lipid & TG (52:1) & 1 & 1 ApoE4 & 0.027 & 0.176 \\
   & Lipid & TG (54:1) & 1 & 1 ApoE4 & 0.031 & 0.203 \\
   & Lipid & TG (50:0) & 1 & 1 ApoE4 & 0.037 & 0.229 \\
   & Organic acid & Malic.acid & 1 & 2 ApoE4 & 0.038 & 0.234 \\
   & Lipid & DG (36:2) & 1 & 1 ApoE4 & 0.039 & 0.235 \\
   & Ox. Stress & PAF (16:0)& 1 & 2 ApoE4 & 0.049 & 0.283 \\ \midrule
  \multirow{10}{*}{SCD} & Lipid & LPC (20:5) & 1 & no ApoE4 & 0.012 & 0.828 \\
   & Lipid & SM (d18:1/23:0) & 1 & 1 ApoE4 & 0.015 & 0.828 \\
   & Aminoacid & L-Tryptophan & 1 & no ApoE4 & 0.024 & 0.828 \\
   & Aminoacid & Putrescine & 1 & 1 ApoE4 & 0.028 & 0.828 \\
   & Aminoacid & Glycine & 1 & 1 ApoE4 & 0.035 & 0.828 \\
   & Lipid & CE (18:2) & 1 & 1 ApoE4 & 0.038 & 0.828 \\
   & Aminoacid & SM (d18:1/22.0) & 1 & 1 ApoE4 & 0.039 & 0.828 \\
   & Ox. Stress & LPA (20:5) & 1 & no ApoE4 & 0.040 & 0.828 \\
   & Lipid & PC (36:3) & 1 & 1 ApoE4 & 0.043 & 0.828 \\
   & Organic acid & Succinic acid & 1 & 1 ApoE4 & 0.043 & 0.828 \\
   & Amine & Citrulline & 1 & 1 ApoE4 & 0.043 & 0.828 \\   \bottomrule  
  \end{tabular}
  \begin{tablenotes}
    \item[] CE: Cholesteryl ester, DG: Diglyceride, LPA: Lyso-sphingolipid LPC: Lysophosphatidylcholine, PAF: Platelet activating factor, SM: Sphingomyelin, TG: Triglyceride
  \end{tablenotes}
\end{threeparttable}}
  \end{table}


\begin{figure}[H]
    \includegraphics[width=0.75\textwidth]{figures/gt2.png}
      \caption{Covariates plot showing the metabolites affected by the number of ApoE4 alleles in AD}
    \label{plot:gt}
  \end{figure}
\clearpage  
\subsubsection{Nested Linear Models}
Several metabolites from all classes were altered by ApoE4 dose, both in AD and SCD. However, after controlling FDR, none of the effects remain significant at $\alpha=0.05$. Among AD patients, metabolites showing trends of a positive effect were several triglycerides, diglycerides, putrescine, 2-ketoglutraric acid, lysophosphatidylcholin, platelet activating factor (16:0) and lyso-phosphatidic acid (18:0) (Table \ref{tab:nested}). Among individuals with SCD, lipid metabolites were not affected as much as in the AD group, with only two sphingomyelin species showing a difference. Aminoacids L-serine, tryptophan, glycine, trytptophan, L-homoserine, putrescine showed trends of effect in this group. L-Tryptophan is negatively associated with ApoE4 dose (at 1x and 2x ApoE4), while L-serine, glycine and L-homoserine are negatively associated only with ApoE4 homozygotes. (Table \ref{tab:nested}). 
\begin{table}[H]
\caption{ApoE4 dose effects on serum metabolites in AD and SCD: results from nested linear model comparison. Full model: clinical background variables and number of ApoE4 alleles, Nested model: clinical background variables only. Colour range green-red reflects trends in positive-negative correlations between ApoE4 dose and metabolites.}
\label{tab:nested}
\centering
\begin{threeparttable}
  \begin{tabular}{clrrrrrrrr} \toprule
    \multicolumn{1}{l}{} & \multicolumn{1}{c}{\multirow{2}{*}{Metabolite}} & \multicolumn{1}{c}{\multirow{2}{*}{P($>$F)}} & \multicolumn{1}{c}{\multirow{2}{*}{FDR}} & \multicolumn{2}{c}{No $\varepsilon_4$} & \multicolumn{2}{c}{1x $\varepsilon_4$} & \multicolumn{2}{c}{2x $\varepsilon_4$} \\
\multicolumn{1}{l}{} & \multicolumn{1}{c}{} & \multicolumn{1}{c}{} & \multicolumn{1}{c}{} & \multicolumn{1}{l}{Coef.} & \multicolumn{1}{l}{P($>$t)$^\ast$} & \multicolumn{1}{l}{Coef.} & \multicolumn{1}{l}{P($>$t)$^\ast$} & \multicolumn{1}{l}{Coef.} & \multicolumn{1}{l}{P($>$t)$^\ast$} \\ \midrule
  \multirow{19}{*}{AD} & Lip.TG (52:3) & 0.001 & 0.156 & {\cellcolor[rgb]{0.98,0.8,0.808}}-1.4 & 0.818 & {\cellcolor[rgb]{0.949,0.973,0.969}}2.6 & 0.004 & {\cellcolor[rgb]{0.929,0.965,0.949}}3.7 & 0.001 \\
   & Lip.TG (52:4) & 0.003 & 0.156 & {\cellcolor[rgb]{0.984,0.988,0.996}}0.6 & 0.888 & {\cellcolor[rgb]{0.969,0.98,0.98}}1.6 & 0.008 & {\cellcolor[rgb]{0.953,0.976,0.973}}2.4 & 0.002 \\
   & DG (36:3) & 0.002 & 0.156 & {\cellcolor[rgb]{0.984,0.957,0.969}}0.0 & 0.631 & {\cellcolor[rgb]{0.984,0.957,0.965}}0.0 & 0.012 & {\cellcolor[rgb]{0.984,0.957,0.969}}0.0 & 0.001 \\
   & OS.HpH.PAF (16:0) & 0.002 & 0.156 & {\cellcolor[rgb]{0.388,0.745,0.482}}34.8 & $<$0.001 & {\cellcolor[rgb]{0.949,0.973,0.969}}2.6 & 0.017 & {\cellcolor[rgb]{0.914,0.961,0.937}}4.6 & 0.001 \\
   & Lip.TG (52:2) & 0.006 & 0.255 & {\cellcolor[rgb]{0.973,0.412,0.42}}-5.1 & 0.469 & {\cellcolor[rgb]{0.957,0.976,0.976}}2.1 & 0.03 & {\cellcolor[rgb]{0.929,0.965,0.949}}3.7 & 0.002 \\
   & Lip.TG (54:5) & 0.01 & 0.373 & {\cellcolor[rgb]{0.965,0.98,0.98}}1.9 & 0.496 & {\cellcolor[rgb]{0.98,0.984,0.992}}0.9 & 0.016 & {\cellcolor[rgb]{0.973,0.984,0.988}}1.3 & 0.006 \\
   & Lip.TG (50:2) & 0.012 & 0.398 & {\cellcolor[rgb]{0.973,0.471,0.478}}-4.5 & 0.297 & {\cellcolor[rgb]{0.98,0.984,0.992}}0.9 & 0.141 & {\cellcolor[rgb]{0.957,0.976,0.973}}2.2 & 0.003 \\
   & Lip.TG (50:1) & 0.018 & 0.45 & {\cellcolor[rgb]{0.98,0.706,0.714}}-2.3 & 0.539 & {\cellcolor[rgb]{0.984,0.988,0.996}}0.7 & 0.179 & {\cellcolor[rgb]{0.965,0.98,0.98}}1.8 & 0.005 \\
   & Lip.TG (54:4) & 0.02 & 0.45 & {\cellcolor[rgb]{0.937,0.969,0.953}}3.5 & 0.318 & {\cellcolor[rgb]{0.976,0.984,0.988}}1.2 & 0.015 & {\cellcolor[rgb]{0.973,0.984,0.984}}1.4 & 0.02 \\
   & Lip.TG (54:6) & 0.017 & 0.45 & {\cellcolor[rgb]{0.984,0.906,0.918}}-0.4 & 0.795 & {\cellcolor[rgb]{0.988,0.988,1}}0.5 & 0.027 & {\cellcolor[rgb]{0.984,0.988,0.996}}0.7 & 0.009 \\
   & Am.Putrescine & 0.029 & 0.515 & {\cellcolor[rgb]{0.984,0.953,0.965}}0.0 & $<$0.001 & {\cellcolor[rgb]{0.984,0.953,0.965}}0.0 & 0.035 & {\cellcolor[rgb]{0.984,0.953,0.965}}0.0 & 0.017 \\
   & OA.2.ketoglutaric.acid & 0.026 & 0.515 & {\cellcolor[rgb]{0.984,0.953,0.965}}0.0 & 0.366 & {\cellcolor[rgb]{0.984,0.953,0.965}}0.0 & 0.953 & {\cellcolor[rgb]{0.984,0.953,0.965}}0.0 & 0.014 \\
   & Lip.TG (50:3) & 0.028 & 0.515 & {\cellcolor[rgb]{0.98,0.824,0.835}}-1.2 & 0.668 & {\cellcolor[rgb]{0.984,0.988,0.996}}0.6 & 0.127 & {\cellcolor[rgb]{0.973,0.984,0.988}}1.3 & 0.008 \\
   & Lip.TG (52:5) & 0.042 & 0.538 & {\cellcolor[rgb]{0.988,0.988,1}}0.5 & 0.789 & {\cellcolor[rgb]{0.988,0.988,1}}0.4 & 0.134 & {\cellcolor[rgb]{0.984,0.988,0.996}}0.7 & 0.014 \\
   & Lip.TG (56:7) & 0.042 & 0.538 & {\cellcolor[rgb]{0.976,0.698,0.706}}-2.4 & 0.095 & {\cellcolor[rgb]{0.988,0.988,1}}0.5 & 0.015 & {\cellcolor[rgb]{0.988,0.988,1}}0.4 & 0.105 \\
   & Lip.TG (56:8) & 0.044 & 0.538 & {\cellcolor[rgb]{0.98,0.804,0.816}}-1.4 & 0.055 & {\cellcolor[rgb]{0.984,0.98,0.992}}0.2 & 0.014 & {\cellcolor[rgb]{0.984,0.973,0.984}}0.2 & 0.151 \\
   & Lip.TG (58:9) & 0.042 & 0.538 & {\cellcolor[rgb]{0.984,0.906,0.918}}-0.4 & 0.068 & {\cellcolor[rgb]{0.984,0.961,0.973}}0.1 & 0.013 & {\cellcolor[rgb]{0.984,0.957,0.969}}0.0 & 0.41 \\
   & LPC (16:0) & 0.033 & 0.538 & {\cellcolor[rgb]{0.922,0.961,0.945}}4.2 & 0.028 & {\cellcolor[rgb]{0.988,0.988,1}}0.3 & 0.237 & {\cellcolor[rgb]{0.98,0.988,0.992}}0.9 & 0.009 \\
   & OS.HpH.LPA (18:0) & 0.044 & 0.538 & {\cellcolor[rgb]{0.988,0.988,1}}0.5 & 0.038 & {\cellcolor[rgb]{0.984,0.957,0.969}}0.0 & 0.178 & {\cellcolor[rgb]{0.984,0.965,0.976}}0.1 & 0.013 \\ \midrule
  \multirow{7}{*}{SCD} & Am.L-Serine & 0.002 & 0.285 & {\cellcolor[rgb]{0.914,0.961,0.937}}4.7 & $<$0.001 & {\cellcolor[rgb]{0.984,0.973,0.984}}0.2 & 0.115 & {\cellcolor[rgb]{0.98,0.839,0.851}}-1.1 & 0.003 \\
   & Am.L-Tryptophan & 0.002 & 0.285 & {\cellcolor[rgb]{0.929,0.965,0.949}}3.7 & 0.004 & {\cellcolor[rgb]{0.984,0.922,0.933}}-0.3 & 0.047 & {\cellcolor[rgb]{0.98,0.808,0.82}}-1.4 & 0.002 \\
   & Am.Glycine & 0.006 & 0.45 & {\cellcolor[rgb]{0.922,0.961,0.941}}4.3 & 0.001 & {\cellcolor[rgb]{0.988,0.988,1}}0.4 & 0.015 & {\cellcolor[rgb]{0.984,0.859,0.871}}-0.9 & 0.052 \\
   & Am.L-homoserine & 0.047 & 0.931 & {\cellcolor[rgb]{0.984,0.953,0.965}}0.0 & 0.001 & {\cellcolor[rgb]{0.984,0.953,0.965}}0.0 & 0.723 & {\cellcolor[rgb]{0.984,0.953,0.965}}0.0 & 0.014 \\
   & Am.Putrescine & 0.045 & 0.931 & {\cellcolor[rgb]{0.984,0.953,0.965}}0.0 & 0.969 & {\cellcolor[rgb]{0.984,0.953,0.965}}0.0 & 0.013 & {\cellcolor[rgb]{0.984,0.953,0.965}}0.0 & 0.927 \\
   & Lip.SM (d18:1/22:0) & 0.043 & 0.931 & {\cellcolor[rgb]{0.937,0.969,0.957}}3.3 & 0.002 & {\cellcolor[rgb]{0.984,0.984,0.996}}0.3 & 0.015 & {\cellcolor[rgb]{0.984,0.98,0.992}}0.3 & 0.448 \\
   & Lip.SM (d18:1/23:0) & 0.029 & 0.931 & {\cellcolor[rgb]{0.973,0.984,0.988}}1.2 & 0.01 & {\cellcolor[rgb]{0.984,0.969,0.98}}0.1 & 0.012 & {\cellcolor[rgb]{0.984,0.973,0.984}}0.2 & 0.277 \\ \bottomrule
\end{tabular}
\begin{tablenotes}
  \item[$\ast$] Not corrected for multiple testing.
  \item[] Am: Aminoacid, Lip: Lipid, DG: Diglyceride, LPA: Lyso-sphingolipid LPC: Lysophosphatidylcholine, SM: Sphingomyelin, TG: Triglyceride, OA: Organic Acid, OS: Oxidative Stress compound, PAF: Platelet activating factor
\end{tablenotes}
\end{threeparttable}
\end{table}
\newpage
\subsection{Classification of ApoE4 and AD status}
All classification models had a multi-class ROC \acrshort{auc} above 0.8 and accuracies significantly higher than the No-Information Rate (p$<2.2e^{-16}$) (Table \ref{tab:clin}) in predicting AD without ApoE4, AD with at least 1 ApoE4, SCD without ApoE4, SCD with at least 1 ApoE4. The worst-performing model was  Multinomial Logistic Regression fitting the clinical data only, while the best-performing one was XGBoost fitting the clinical data and the 6 meta-metabolites. Adding serum metabolite information (either the full 230-metabolite matrix or its 6-factor projection) seems to slightly increase the discriminatory power of the models.
Notably, looking at the individual ROC curves (Fig. \ref{roc:full}, \ref{roc:6mlogit} and Appendix \ref{appendixA}), as well as the confusion matrices (Fig. \ref{cm:full} and \ref{cm:xgboost}) all models were able to discriminate better among certain classes (AD+E4 vs. SCD+E4, AD+E4 vs. SCD, AD-E4 vs. SCD+E4 and AD-E4 vs. SCD-E4) compared to others (AD+E4 vs. AD-E4 and SCD+E4 vs. SCD-E4).
\begin{table}[H]
  \centering
\caption{Performance metrics of multi-class classification of ApoE4 and AD status (AD without ApoE4, AD with at least 1 ApoE4, SCD without ApoE4, SCD with at least 1 ApoE4), obtained from 10-fold CV repeated 100 times.} 
\label{tab:class_results}
\begin{threeparttable}
\begin{tabular}{clllll}\toprule
Model  & Fitting & AUC & Accuracy (95/\% CI) & P(NIR$>$Acc.) & NIR \\ \midrule
\multirow{3}{*}{MLR} & Clinical features only & 0.814 & 0.4807 (0.4744, 0.4869) & $<2.2e^{-16}$ & \multirow{5}{*}{0.3522} \\
& Clinical features + 230 metabolites & 0.834 & 0.5753 (0.5692, 0.5815) & $<2.2e^{-16}$ &  \\
& \multirow{3}{*}{Clinical features + 6 latent factors} & 0.818 & 0.5313 (0.525, 0.5375) & $<2.2e^{-16}$ &  \\
Decision Tree & & 0.818 & 0.5082 (0.502, 0.5145) & $<2.2e^{-16}$ &  \\
XGBoost & & 0.836 & 0.5944 (0.5882, 0.6005) & $<2.2e^{-16}$ &  \\ \bottomrule
\end{tabular}
\begin{tablenotes}
  \item[]  AUC: Area Under the (ROC) Curve, MLR: Multinomial Logistic Regression, NIR: No-Information Rate
\end{tablenotes}
\end{threeparttable}
\end{table}

\begin{figure}[h]
  \includesvg[width=0.6\linewidth]{figures/fullmultinom.svg}
  \caption{\label{roc:full}ROC curves of Multinomial Logistic Regression fitting the 230 metabolites on top of clinical background variables, obtained from repeated (100 times) 10-fold CV. AD: AD without ApoE4, ADE4: AD with at least 1 ApoE4, SCD: SCD without ApoE4, SCDE4: SCD with at least 1 ApoE4}
  
\end{figure} \clearpage
\begin{figure}[ht]
\includesvg[width=0.6\linewidth]{figures/6factor.svg}
\caption{\label{roc:6mlogit} ROC curves of Multinomial Logistic Regression fitting the 6 ML-estimated latent factors on top of the clinical background variables, obtained from repeated (100 times) 10-fold CV. AD: AD without ApoE4, ADE4: AD with at least 1 ApoE4, SCD: SCD without ApoE4, SCDE4: SCD with at least 1 ApoE4}
\end{figure} 
\begin{figure}[h]
  \includesvg[width=0.7\linewidth]{figures/cm_mlr.svg}
  \caption{Confusion matrix showing the true (Target) and predicted (Prediction) classes of Multinomial Logistic Regression fitting the 230 metabolites on top of clinical background variables, obtained from repeated (100 times) 10-fold CV. AD: AD without ApoE4, ADE4: AD with at least 1 ApoE4, SCD: SCD without ApoE4, SCDE4: SCD with at least 1 ApoE4.}
  \label{cm:full}
\end{figure}
\begin{figure}[ht]
    \includesvg[width=0.7\linewidth]{figures/cm_xgb.svg}
    \caption{Confusion matrix showing the true (Target) and predicted (Prediction) classes of XGBoost fitting the 6 ML-estimated latent factors on top of the clinical background variables, obtained from repeated (100 times) 10-fold CV. AD: AD without ApoE4, ADE4: AD with at least 1 ApoE4, SCD: SCD without ApoE4, SCDE4: SCD with at least 1 ApoE4}
  \label{cm:xgboost}
\end{figure}
\begin{table}[H] 
	\centering
	\caption{\label{tab:fimp} Features with the highest feature importance scores in predicting AD without ApoE4, AD with at least 1 ApoE4, SCD without ApoE4 or SCD with at least 1 ApoE4 with a Multinomial Logistic Regression model fitting clinical background data and 230 metabolites.}
	\begin{tabular}{clc}
		\toprule
      Class &	Feature & Overall (\%) \\ \midrule
    Amine & Putrescine            & 100.00 \\
    Ox. stress & HpH.Spha.1.P.C18.0    & 97.43  \\
    Organic acid & Uracil         & 91.78  \\
    Lipid & TG (56:0)             & 88.86  \\
    Organic acid & 3.Hydroxybutyric.acid & 87.29  \\
    Clinical &    Cholesterol medication   & 84.54  \\
    Amine & Sarcosine             & 83.54  \\
    Lipid & PE.O. (38:5)           & 81.60  \\
    Ox. stress & HpH.LPA.C20.5         & 72.43  \\
    Amine & L-Tryptophan          & 71.34  \\
    Lipid & SM (d18:1/20:1)       & 70.52  \\
    Amino acid & Glutathione           & 69.96  \\
    Amino acid & L-Serine              & 69.49  \\
    Amino acid & L-Histidine           & 69.41  \\
    Ox. stress & HpH.LPA.C18.0         & 68.81  \\
    Amino acid & Histamine             & 67.99  \\
    Amino acid & L-Threonine           & 67.85  \\
    Amino acid & L-Glutamine           & 67.70  \\
    Lipid & PC (38:3)             & 65.30  \\
    Organic acid & Pyruvic.acid          & 64.96 \\ \bottomrule
	\end{tabular}
\end{table}
The most important metabolites in identifying the ApoE4AD phenotypes were putrescine, Spha.1.P-(18:0), uracil, triglyceride (56:0), 3- hydroxybutyric acid, sarcosine, phosphoethanolamine (38:5), LPA (20:5), L-trypthophan, sphingomyelin (d18:1/20:1), glutathione, L-serine, L-histidine, LPA (18:0), histamine, L-threonine, L-glutamine, phosphatidylcholine (38:3), see Table \ref{tab:fimp}. 

\subsection{Metabolite Covariance Network Analysis}
The Gaussian graphical modelling between ApoE4 carriers and non-carriers with AD showed distinct correlations between the metabolites (Fig. \ref{netAD}). The ApoE4 carriers' network was less sparse, having less edges compared to the non-carriers'. Moreover, the communities of covarying metabolites are distinct among the two groups, as shown in Fig. \ref{comms}.
\begin{figure}[!hb]
\centerline{\includesvg[width=1.2\textwidth]{figures/NetAD.svg}}
\caption{\label{netAD} Metabolite covariance network topologies among ApoE4 non-carriers (top-left), carriers (top-right) and differential edge network (bottom) in AD.}  
\end{figure}

The Wilcoxon Signed Rank Test showed higher degrees of centrality in ApoE4 carriers compaired to non-carriers (p = 0.001). In the ApoE4 positive group, central metabolites were 1-methylhistidine, citrulline and pyroglutamic acid, followed by asymmetric dimethylarginine (ADMA), symmetric dimethylarginine (SDMA), 2-ketoglutaric acid, uracil, histamine, L-arginine, 5 organic acids, 3 lipids and 2 oxidative stress compounds (Table \ref{tab:degrees}). ApoE4 non-carriers had 3-methylhistidine as top central metabolite, followed by lyso-sphingolipid 1-P(18:1), methoxytyrosine, 3-aminoisobutyric acid, dopamine, putrescine, glutamate and succinate, sphingomyelin (d18:1/21:0) and oxidative stress compound cLPA (18:2).

\begin{table}[H]
  \caption{\label{tab:degrees} Central (centrality degreee $>$5) metabolites in ApoE4 carriers and non-carriers with AD.}
  \begin{threeparttable}
  \begin{tabular}{lclc}\toprule
  \multicolumn{2}{c}{ApoE4 carriers} & \multicolumn{2}{c}{ApoE4 non-carriers} \\
  Metabolite & Degree & Metabolite & Degree \\ \midrule
  Am.X1.Methylhistidine & 7 & Am.X3.Methylhistidine & 7 \\
  Am.Citrulline & 7 & Lip.SM.d18.1.20.0. & 6 \\
  OA.Pyroglutamic.acid & 7 & Am.X3.Methoxytyrosine & 5 \\
  Am.ADMA & 6 & Am.DL.3.aminoisobutyric.acid & 5 \\
  Am.SDMA & 6 & Am.Dopamine & 5 \\
  OA.2.ketoglutaric.acid & 6 & Am.Putrescine & 5 \\
  OA.Uracil & 6 & OA.Glutamic.Acid & 5 \\
  Am.Histamine & 5 & OA.Succinic.acid & 5 \\
  Am.L.Arginine & 5 & Lip.SM.d18.1.21.0. & 5 \\
  Am.L.Kynurenine & 5 & OS.HpH.cLPA.C18.2 & 5 \\
  Am.N6.N6.N6.Trimethyl.L.lysine & 5 &  &  \\
  Am.Putrescine & 5 &  &  \\
  OA.Lactic.acid & 5 &  &  \\
  OA.Fumaric.acid & 5 &  &  \\
  OA.Aspartic.acid & 5 &  &  \\
  OA.Iminodiacetate & 5 &  &  \\
  OA.3.hydroxyisovaleric.acid & 5 &  &  \\
  Lip.PC.36.5. & 5 &  &  \\
  Lip.SM.d18.1.21.0. & 5 &  &  \\
  Lip.SM.d18.1.23.0. & 5 &  &  \\
  OS.HpH.cLPA.C18.2 & 5 &  &  \\
  OS.cHpH.cLPA.C20.3 & 5 &  & \\ \bottomrule
  \end{tabular}
\begin{tablenotes}
\item[] ADMA: Asymmetric Dimethilarginine, SD, Am: Aminoacid, Lip: Lipid, DG: Diglyceride, LPA: Lyso-sphingolipid , SM: Sphingomyelin, TG: Triglyceride, OA: Organic Acid, OS: Oxidative Stress compound, PC: Phosphatidylcholine, SDMA: Symmetric Dimethilarginine
  
\end{tablenotes}
\end{threeparttable}
  \end{table}

\begin{figure}
  \includesvg[width=0.7\textwidth]{figures/comms.svg}
  \caption{\label{comms} Communities of (densely correlated) metabolites in ApoE4 non-carriers (top) and ApoE4 carriers with AD (bottom).}
\end{figure}

\clearpage
\section{Discussion} \label{discuss}
This study had two objectives: to unveil potentially unknown pathways of systemic metabolic deregulation attributed to ApoE4 (dose or presence), and to propose a comprehensive methodology for high-dimensional data analysis in the context of AD. To this end, serum metabolic signatures were assessed among SCD and AD with three approaches: differential expression, multiclass classification and network analysis. Differential expression highlighted metabolites whose levels are shifted by ApoE4 dose. The multiclasss classification reveals metabolites that, on top of clinical background variables, can be fitted to predict AD and ApoE4 status. Covariance network analysis reveals metabolic interdependencies among ApoE4 carriers and non-carriers in AD. The serum metabolites reported here might serve as a \textit{proxy} of metabolic perturbations in the brain. The metabolites were measured in serum obtained from peripheral blood, an (almost) non-invasive alternative to spinal aspiration for \acrshort{csf}.

The metabolites that emerged from the analysis were generally different between the approaches, and among AD-SCD. Nevertheless, certain metabolites appeared in all, or many of the metabolic signatures. 

A metabolite that appeared in all three metabolic signatures was putrescine: trends of positive ApoE4 dose effects in AD, top metabolite in determining ApoE4 and AD status and central metabolite (5 dependent metabolites regardless of ApoE4 status) in AD. Putrescine appeared increased in a recent study \cite{Ju2021AstrocyticUC}, where it is described as a toxic by-product of activated urea cycle in A$\beta$-reactive astrocytes of AD brains \cite*{Ju2021AstrocyticUC,Wong2022PathogenicP}.

Trends of decreased putrescine and four amino acids with increasing ApoE4 presence were observed in the SCD group: L-serine, L-tryptophan, glycine and L-homoserine. Tryptophan and serine proved also important in predicting ApoE4 and AD status. In the brain, tryptophan is catabolised to kynurenine by the enzymes indoleamine 2,3-dioxygenase and tryptophan 2,3-dioxygenase, through the kynurenine pathway \cite*{Liang2022KynureninePM}. In this study, L-kynurenine was found central in the serum metabolome, branched with five  metabolites. Recent reviews discussed kurenine pathway metabolites to be closely associated with AD pathogenesis \cite{Liang2022KynureninePM,Sharma2022KynurenineMA}. Increased kynurenine pathway metabolites may reflect the extent of neuroinflammation in amyotrophic lateral sclerosis, frontotemporal dementia and early onset AD \cite*{Heylen2023BrainKP}. A whole-blood targeted metabolomics panel also revealed increased kynurenine pathway metabolites \cite{Teruya2021WholebloodMO}. Conversely, L-serine levels appear to be deficient in AD brains due to impaired glycolysis \cite{LeDouce2020ImpairmentOG}. 

Ketoglutarate showed trends of decreased levels with increasing ApoE4 presence in the global test and the ANCOVA F-tests exclusively in the AD group. The network analysis revealed it is also a central metabolite, driving the variance of six other metabolites. In the brain, the $\alpha$-ketoglutarate dehydrogenase complex is perceived as a "hub of plasticity in neurodegeneration and regeneration", as it reflects the impaired glucose metabolism and key enzymes of the tricarboxylic acid cycle \cite*{Hansen2022TheD}.

\leavevmode\newline \textbf{ApoE4 dose effects on serum metabolite levels}\hspace{.25cm}
The global test for ApoE4 dose effects on serum metabolite levels revealed positive dose effects on lipids in AD, mainly triglycerides and diglycerides. This is consistent with evidence from Caucasian \cite{Maxwell2013APOEMT,CARVALHOWELLS20121447,Bernath2020SerumTI} and southern Chinese populations \cite{Gan2022EffectsPopulation} that shows elevated triglycerides associated with ApoE4. The same test was not significant in the SCD group; mostly amines and amino acids showed trends of an effect. This could putatively reflect compensatory mechanisms still in place for the debilitating effects of ApoE4.

The nested linear model comparison did not reveal any significant ApoE4 dose effects, after correcting FDR. Even though the methods are fundamentally different, this can be partially explained considering the covariates in each approach. The global test was corrected for sex only, while the nested models fitted several background clinical factors. That is, the metabolic variance explained by ApoE4 dose in the global test might be partially attributed to other clinical factors in the nested models, such as hypercholesterolemia. Notably, the metabolites showing trends of an ApoE4 dose effect were different between the AD and SCD group. Consistent with the global test, AD group presented mainly a lipid (tri- and diglycerides) signature, while the SCD group an amino one (serine, tryptophan, glycine, homoserine and putrescine).

\leavevmode\newline \textbf{Classification of ApoE4 and AD status}\hspace{.25cm}
Serum metabolic information, in the form of a semi-targeted metabolomic panel or its latent 6-factor projection seems to slightly increase the classification potential of the models. As expected, \acrshort{xgb} performed better than the multinomial logistic regression on the same data. Interestingly, all models discriminated better among certain classes (AD+E4 vs. SCD+E4, AD+E4 vs. SCD, AD-E4 vs. SCD+E4 and AD-E4 vs. SCD-E4) compared to others (AD+E4 vs. AD-E4 and SCD+E4 vs. SCD-E4). This can be putatively attributed to the class imbalance between AD+E4 (n = 34) and AD (n = 86), as well as SCD+E4 (n = 87) and SCD (n = 40). It might also show that the metabolic nuances among ApoE4 carriers and non-carriers are not as pronounced as among AD and SCD.

Each of the top four metabolites in delineating ApoE4 and AD status belonged to one of the major metabolite classes studied: amines (putrescine), oxidative stress compounds (sphingosine-1-phosphate (C18:0)), organic acids (uracil) and lipids (TG (56:0)). Perturbations in sphingolipid metabolism are observed in AD \cite{mielke2010alterations}, even in the stage of MCI \cite{den2023sphingolipids}. \Citeauthor{den2023sphingolipids} reported increased sphingosine-1-phosphate in CSF and plasma of homozygous ApoE4 carriers \cite{den2023sphingolipids}. Another recent study highlights sphingolipids and sphingosine-1-phosphate as diagnostic markers for AD \cite{d2022sphingolipid}.

\leavevmode\newline \textbf{Network analysis}\hspace{.25cm}
The metabolite covariance network of ApoE4 carriers was more random and less cohesive compared to ApoE4 non-carriers, in line with \citeauthor{deLeeuw2017Blood-basedDisease}'s findings. Amines and amino acids are the top central metabolites in both groups, followed by organic acids. 1-methylhistidine was the top central metabolite in ApoE4 carriers, while 3-methylhistidine in ApoE4 non-carriers. Both compounds are produced by methylation of the amino acid histidine at different positions. A recent metabolomics analysis revealed disturbances in methylhistidine metabolism in AD \cite{kalecky2022targeted}.

\leavevmode\newline \textbf{Strengths and Limitations}\hspace{.25cm} A strength of this study is the efficient comprehensive methodology it presents for omics data analysis in AD research. The employed statistical methods here comprise a state-of-the-art analytical framework for metabolomics (or any -omics) data to extract insights or assist diagnostic processes. For instance, multi-class classification models predicting ApoE4 and AD status performed on par, either fitting the 230 metabolites or six meta-metabolites accounting for $\sim$30\% of their variance, obtained with \textsf{FMradio}. Thus, the latter is reported as a powerful package for dimension reduction, that still retains important variance in the data, even with very low-dimension outputs.

An important limitation is that metabolites were measured in serum -rather than CSF- which only reveals metabolic perturbations at systemic level. Further, AD is a complex and multifaceted disorder with several proposed pathophysiological mechanisms. Hence, focusing on links between ApoE4 (status or dose) and serum metabolome in AD provides a limited view of the underlying pathologies. In this sense, gaining insights in ApoE genotype effects on serum metabolites might prove more informative. Nevertheless, the relative genotype counts did not allow further analysis without grouping. Additionally, the significantly increased tri- and diglycerides in the global test might be attributed to hypertriglyceremia (that often coexists with hypercholesterolemia) and was not measured in this study. Moreover, it is important to note that the SCD group in the data is not a \textit{control} group, as subjects with varying degrees of cognitive impairment were included, who might develop AD or another neurodegenerative disease later.

\leavevmode\newline \textbf{Future research}\hspace{.25cm}

\clearpage
\section{Conclusion} \label{concl}


%--------------- Main matter -----------------------------------------
%-------------------------------------------------------------------


%--------------------------------------------------------------------
%--------------- References -----------------------------------------
\newpage
\section*{References}
\printbibliography[heading=none]
\clearpage
%--------------- References -----------------------------------------
%--------------------------------------------------------------------
\appendix 
\clearpage
\section{Multiclass ROC curves} \label{appendixA}
\begin{figure}[htb]
  \includesvg[width=0.75\textwidth]{figures/bench.svg}
  \caption{ROC curves of benchmark model: Multinomial Logistic Regression fitting the clinical background features only, obtained from repeated (100 times) 10-fold CV.}
  \label{roc:bench}
\end{figure}
\begin{figure}[htb]
\includesvg[width=.75\textwidth]{figures/tree.svg}
\caption{ROC curves of Decision Tree fitting the 6 ML-estimated latent factors on top of the clinical background variables, obtained from repeated (100 times) 10-fold CV}
\label{roc:tree}
\end{figure}
\begin{figure}
\includesvg[width=.75\textwidth]{figures/xgb.svg}
\caption{ROC curves of Xtreme Gradient Booster fitting the 6 ML-estimated latent factors on top of the clinical background variables, obtained from repeated (100 times) 10-fold CV}
\label{roc:xgb}
\end{figure}

\clearpage
\section{R Session Information} \label{appendixB}

\begin{verbatim}
R version 4.3.2 (2023-10-31)
Platform: aarch64-apple-darwin20 (64-bit)
Running under: macOS Sonoma 14.2.1

Matrix products: default
BLAS:   .../vecLib.framework/Versions/A/libBLAS.dylib 
LAPACK: LAPACK version 3.11.0

locale:
[1] en_US.UTF-8

time zone: Europe/Amsterdam
tzcode source: internal

attached base packages:
[1] stats     graphics  grDevices datasets  utils     methods   base     

other attached packages:
 [1] rags2ridges_2.2.7       nnet_7.3-19             DMwR2_0.0.2                       
 [4] dplyr_1.1.4             caret_6.0-94            lattice_0.21-9                    
 [7] globaltest_5.56.0       survival_3.5-7          heatmaply_1.5.0               
[10] xgboost_1.7.6.1         pROC_1.18.5             FMradio_1.1.1       
[13] future_1.33.0           ggthemes_5.0.0          corpcor_1.6.10
[16] viridisLite_0.4.2       plotly_4.10.3           ggplot2_3.4.4
[19] rpart_4.1.21            furrr_0.3.1             viridis_0.6.4 
[22] e1071_1.7-14            

loaded via a namespace (and not attached):
 [1] splines_4.3.2           bitops_1.0-7            tibble_3.2 1                              
 [4] XML_3.99-0.16           lifecycle_1.0.4         globals_0.16.2                            
 [7] magrittr_2.0.3          Hmisc_5.1-1             rmarkdown_2.25                        
[10] lubridate_1.9.3         zlibbioc_1.48.0         sfsmisc_1.1-16                                 
[13] RCurl_1.98-1.13         ipred_0.9-14            lava_1.7.3                               
[16] S4Vectors_0.40.2        listenv_0.9.0           gRbase_2.0.1                                
[19] codetools_0.2-19        tidyselect_1.2.0        TSP_1.2-4                                             
[22] jsonlite_1.8.8          Formula_1.2-5           iterators_1.0.14                     
[25] snowfall_1.84-6.3       Rcpp_1.0.11             glue_1.6.2                           
[28] tufte_0.13              TTR_0.24.4              GenomeInfoDb_1.38.2         
[31] fastmap_1.1.1           fansi_1.0.6             digest_0.6.33                  
[34] RSQLite_2.3.4           utf8_1.2.4              tidyr_1.3.0                  
[37] recipes_1.0.9           class_7.3-22            httr_1.4.7                      
[40] gtable_0.3.4            timeDate_4032.109       blob_1.2.4                     
[43] RBGL_1.78.0             GSEABase_1.64.0         scales_1.3.0                        
[46] knitr_1.45              rstudioapi_0.15.0       tzdb_0.4.0                         
[49] curl_5.2.0              proxy_0.4-27            cachem_1.0.8                     
[52] foreign_0.8-85          AnnotationDbi_1.64.1    pillar_1.9.0                        
[55] VGAM_1.1-9              xtable_1.8-4            cluster_2.1.4                       
 [58] cli_3.6.2               compiler_4.3.2          rlang_1.1.2                     
 [61] plyr_1.8.9              stringi_1.8.3           assertthat_0.2.1               
 [64] Matrix_1.6-1.1          hms_1.1.3               bit64_4.0.5                      
 [67] quantmod_0.4.25         bit_4.0.5               hardhat_1.3.0 
 [70] graph_1.80.0            xts_0.13.1              MASS_7.3-60 
 [73] dendextend_1.17.1       backports_1.4.1         yaml_2.3.8   
 [76] DBI_1.1.3               RColorBrewer_1.1-3      expm_0.999-8   
 [79] purrr_1.0.2             BiocGenerics_0.48.1     seriation_1.5.4 
 [82] IRanges_2.36.0          RSpectra_0.16-1         GenomeInfoDbData_1.2.11
 [85] annotate_1.80.0         parallelly_1.36.0       stats4_4.3.2 
 [88] foreach_1.5.2           tools_4.3.2             snow_0.4-4
 [91] prodlim_2023.08.28      gridExtra_2.3           xfun_0.41 
 [94] ca_0.71.1               withr_2.5.2             BiocManager_1.30.22
 [97] timechange_0.2.0        R6_2.5.1                colorspace_2.1-0
[100] generics_0.1.3          renv_1.0.3              data.table_1.14.10
[103] htmlwidgets_1.6.4       ModelMetrics_1.2.2.2    pkgconfig_2.0.3
[106] registry_0.5-1          XVector_0.42.0          htmltools_0.5.7
[109] Biobase_2.62.0          png_0.1-8               gower_1.0.1
[112] reshape2_1.4.4          checkmate_2.3.1         nlme_3.1-163
[115] zoo_1.8-12              stringr_1.5.1           parallel_4.3.2
[118] grid_4.3.2              reshape_0.8.9           vctrs_0.6.5
[121] htmlTable_2.4.2         evaluate_0.23           readr_2.1.4
[124] crayon_1.5.2            future.apply_1.11.0     fdrtool_1.2.17 
[127] munsell_0.5.0           Biostrings_2.70.1       lazyeval_0.2.2 
[130] KEGGREST_1.42.0         igraph_1.6.0            memoise_2.0.1 
[133] base64enc_0.1-3         webshot_0.5.5
\end{verbatim}



\end{document}



% This is a template for a thesis at Biometris created by C.F.W. Peeters in 2021 